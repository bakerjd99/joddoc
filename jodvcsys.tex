\section{JOD and Version Control Systems}\label{ap:jodvcsys}

Despite JOD's backup and restore facilities, see \texttt{bnl}, \texttt{bget}, \texttt{packd} and
\texttt{restd} on pages \pageref{ss:bnl}, \pageref{ss:bget}, \pageref{ss:packd} and \pageref{ss:restd}, JOD is not 
a  source code version control system like \href{http://git-scm.com/}{\texttt{Git}}\index{version control!\texttt{Git}} \cite{gitsite} or 
\href{http://subversion.tigris.org/}{\texttt{Subversion}}\index{version control!\texttt{Subversion}} \cite{subvsite}. 
 JOD's primary
purpose is efficiently  refactoring, shuffling and recombining J words not tracking their detailed histories. 
\emph{Traditional version control systems focus on the history
of source code} and provide detailed  merge, security and multiuser network 
facilities that JOD  lacks.  However, since JOD generates
standard J source code scripts it's easy to use JOD with version control systems. 
The main difficulty is choosing a suitable level of detail: \emph{dictionary, script} or \emph{word.} 
The following shows how \texttt{Git} can be used for each of these levels. \texttt{Git} has
a number of graphical \texttt{GUI} interfaces these examples use bash shell commands.

\begin{enumerate}
\item \textbf{Dictionary:}\label{it:dictlev} \texttt{make}, see page~\pageref{ss:make}, can dump entire
dictionaries as a single J script. Dump scripts contain all\footnote{Word references 
are not present in dump scripts. They can
be easily regenerated with \texttt{globs}, see page~\pageref{ss:globs}.} dictionary word definitions,
test scripts, groups, suites and macros. Storing dump scripts in version control systems is
an effective and simple way of tracking dictionary changes.  To create dump scripts I run
the macro \texttt{dumpput}.
\texttt{dumpput} is stored in the \texttt{utils} dictionary; it dumps the put dictionary and
copies the generated script to a common local directory and a \href{http://www.dropbox.com/}{Dropbox} \cite{dropbsite}
staging directory that I
use for moving JOD dictionaries between various machines.

\begin{lstlisting}[frame=single,framerule=0pt,basicstyle=\ttfamily\footnotesize]
   NB. Step 1: J session commands - open dictionaries
   od ;:'docs utils' [ 3 od ''
+-+-----------------+----+-----+
|1|opened (rw/ro) ->|docs|utils|
+-+-----------------+----+-----+
   
   1 rm 'dumpput'  NB. run dump macro - (utils) must be on path
+-+---------------------------+------------------------------+
|1|object(s) on path dumped ->|c:/jod/j701/docs/dump/docs.ijs|
+-+---------------------------+------------------------------+
+-----------------------------+
|c:/jod/j701/joddumps/docs.ijs|
+-----------------------------+
+-------------------------------+
|c:/db/Dropbox/joddumps/docs.ijs|
+-------------------------------+
\end{lstlisting}

\begin{lstlisting}[language=bash,frame=single,framerule=0pt
,basicstyle=\ttfamily\footnotesize,backgroundcolor=\color{CodeBackGround}]
$ echo Step 2: Bash shell commands > /dev/null

bakerjd99@NINJA /c/jod/j701/joddumps (master)
$ pwd
/c/jod/j701/joddumps

bakerjd99@NINJA /c/jod/j701/joddumps (master)
$ git status -s
M  docs.ijs
 M joddev.ijs
 M utils.ijs

bakerjd99@NINJA /c/jod/j701/joddumps (master)
$ git commit -m 'recent changes to docs.ijs dictionary'
[master 1577d1a] recent changes to docs.ijs dictionary
 1 files changed, 46 insertions(+), 4 deletions(-)
\end{lstlisting}

\item \textbf{Script:} Word and test scripts generated by JOD are stored
in a dictionary's  \texttt{script} and \texttt{suite} subdirectories, see Figure~\ref{eps:joddirs} on page~\pageref{eps:joddirs}.
In the following a \texttt{Git} repository has been created in the \texttt{script} subdirectory
and the contents of the \texttt{exif} group have been edited and regenerated.

\begin{lstlisting}[frame=single,framerule=0pt,basicstyle=\ttfamily\footnotesize]
   NB. Step 1: J session commands - open dictionaries
   od ;:'smugdev smug image utils' [ 3 od ''
+-+-----------------------+-------+----+-----+-----+
|1|opened (rw/ro/ro/ro) ->|smugdev|smug|image|utils|
+-+-----------------------+-------+----+-----+-----+

   NB. edit (exif) content and save changes ...
   
   NB. regenerate (exif) script
   mls 'exif'
+-+--------------------+-----------------------------------+
|1|load script saved ->|c:/jod/j701/smugdev/script/exif.ijs|
+-+--------------------+-----------------------------------+
\end{lstlisting}

\begin{lstlisting}[language=bash,frame=single,framerule=0pt
,basicstyle=\ttfamily\footnotesize,backgroundcolor=\color{CodeBackGround}]
$ echo Step 2: Bash shell commands > /dev/null

bakerjd99@NINJA /c/jod/j701/smugdev/script (master)
$ pwd
/c/jod/j701/smugdev/script

bakerjd99@NINJA /c/jod/j701/smugdev/script (master)
$ git status -s
 M exif.ijs

bakerjd99@NINJA /c/jod/j701/smugdev/script (master)
$ git add exif.ijs

bakerjd99@NINJA /c/jod/j701/smugdev/script (master)
$ git commit -m '(masspixels) added to description of (exif) interface'
[master e93ffe5] (masspixels) added to description of (exif) interface
 1 files changed, 13 insertions(+), 11 deletions(-)
\end{lstlisting}

\item \textbf{Word:} JOD does not directly generate individual word scripts 
but it is easy to define
a simple utility that does. \texttt{pwf}\footnote{
\texttt{pwf} is not a complete solution to exporting 
individual JOD objects as scripts. It only exports words
 and ignores tests, groups, macros and other objects.
It does not address the issue of synchronizing exported
objects with dictionary state. For example, dictionary word
deletions are not propagated. If you wish to track 
dictionary state use the \textbf{\texttt{Dictionary}} (\pageref{it:dictlev})
level method.}
writes individual JOD put dictionary
word files. \texttt{pwf} is stored in the \texttt{utils} dictionary.

\begin{lstlisting}[frame=single,framerule=0pt,basicstyle=\ttfamily\footnotesize]
pwf=: 3 : 0

NB.*pwf v-- write put dictionary words as script files.
NB.
NB. monad:  pwf clPattern
NB.
NB.   pwf 're'  NB. write put dictionary words with prefix 're'
NB.   pwf ''    NB. write all put dictionary words
NB.
NB. dyad:   clPath pwf clPattern
NB.
NB.   'c:/temp' pwf 'de' NB. write to given directory

'' pwf y
:
NB. JOD references !(*)=. dnl get badrc_ajod_ ok_ajod_ 
NB. !(*)=. isempty_ajod_ jpathsep_ajod_ makedir_ajod_ write_ajod_
pk=.  >@{                        
tsl=. ] , ('\'"_ = {:) }. '\'"_  
if.     badrc_ajod_ ws=. 0 _1 dnl y        do. ws return.
elseif. badrc_ajod_ ws=. 0 10 get 1 pk ws  do. ws return.
NB. individual word scripts using short description text for tacits
elseif. badrc_ajod_ ws=. 0 0 1 wttext__MK__JODobj 1 pk ws  do. ws return.
elseif.do.
  try.
    NB. if (x) path is empty use put dictionary directory (alien\words)
    if. isempty_ajod_ x do.
      DL=. {:{.DPATH__ST__JODobj NB. !(*)=. DL
      NB. insure subdirectory when (x) is empty 
      NB. when (x) is nonempty assume it exists
      makedir_ajod_ <jpathsep_ajod_ tsl x=. ALI__DL,'words'
    end.
    NB. write individual word files
    ws=. 1 pk ws
    wpf=. (<jpathsep_ajod_ tsl x) ,&.> (0 {"1 ws) ,&.> <'.ijs'
    ok_ajod_ wpf [ (toHOST&.> 1 {"1 ws) write_ajod_&.> wpf
  catchd. jderr_ajod_ 'unable to write all word file(s)'
  end.
end.
)
\end{lstlisting}

Using \texttt{pwf} is a simple matter of getting and running it.  The following
exports  \texttt{joddev} words  to \verb|joddev/alien/words| 
and then commits differences in \texttt{Git}.

\begin{lstlisting}[frame=single,framerule=0pt,basicstyle=\ttfamily\footnotesize]
   NB. Step 1: J session commands - open dictionaries
   od ;:'joddev jod utils' [ 3 od ''
+-+--------------------+------+---+-----+
|1|opened (rw/ro/ro) ->|joddev|jod|utils|
+-+--------------------+------+---+-----+

   NB. load (pwf, showpass) into the (ijod) locale
   'ijod' get ;:'pwf showpass'
+-+-----------------+
|1|2 word(s) defined|
+-+-----------------+

   NB. edit/modify/create words and save changes ...

   #showpass pwf ''  NB. write and count word files
+-+------------------------------------------+...
|1|c:/jod/j701/joddev/alien/words/ASCII85.ijs|...
+-+------------------------------------------+...
121
\end{lstlisting}

\begin{lstlisting}[language=bash,frame=single,framerule=0pt
,basicstyle=\ttfamily\footnotesize,backgroundcolor=\color{CodeBackGround}]
$ echo Step 2: Bash shell commands > /dev/null

bakerjd99@NINJA /c/jod/j701/joddev/alien/words (master)
$ pwd
/c/jod/j701/joddev/alien/words

bakerjd99@NINJA /c/jod/j701/joddev/alien/words (master)
$ git status -s
 M pwf.ijs

bakerjd99@NINJA /c/jod/j701/joddev/alien/words (master)
$ git add pwf.ijs

bakerjd99@NINJA /c/jod/j701/smugdev/script (master)
$ git commit -m '(pwf) comments added'
[master e93fga4] (pwf) comments added
 1 files changed, 3 insertions(+), 2 deletions(-)
\end{lstlisting}







\end{enumerate}  