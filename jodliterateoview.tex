% this jodliterate overview
\section{\texttt{jodliterate} Overview}

% this jodliterate overview
%\section{\texttt{jodliterate} Overview}

\texttt{jodliterate} is a J utility script that generates
\href{https://en.wikipedia.org/wiki/Literate_programming}{literate} documents directly from JOD dictionary groups.

See the following for details:
\begin{enumerate}
\item
 \href{https://github.com/bakerjd99/jacks/blob/master/jodliterate/Using%20jodliterate.ipynb}{Using jodliterate notebook}.
\item
 \href{https://github.com/bakerjd99/jacks/blob/master/jodliterate/UsingJodliterate.pdf}{Using jodliterate PDF}.
\end{enumerate}

\subsection{\texttt{jodliterate} Interface}

\begin{Shaded}
\begin{Highlighting}[]
\hyperlink{THISPANDOC}{\NormalTok{THISPANDOC      }}\CommentTok{full pandoc path {-} use (pandoc) if on shell path}
\hyperlink{grplit}{\NormalTok{grplit          }}\CommentTok{make latex for group (y)}
\hyperlink{ifacesection}{\NormalTok{ifacesection    }}\CommentTok{interface section summary string}
\hyperlink{setjodliterate}{\NormalTok{setjodliterate  }}\CommentTok{prepare for processing}
\end{Highlighting}
\end{Shaded}

\subsection{\texttt{jodliterate} and JOD}

\texttt{jodliterate} makes some assumptions about the J code and
document text it processes. It assumes:

\begin{enumerate}[1.]
\item
  All source code is stored in JOD dictionaries.
\item
  \texttt{jodliterate} document fragments are either:
  \begin{itemize}
  \item Markdown group headers: \verb|2 9 disp 'groupname'|
  \item \LaTeX\ suffixed macros: \verb|4 disp 'groupname','_oview_tex'|
  \end{itemize}

% \item
%   Non-J code is confined to the JOD group header\footnote{This
%     limitation may be eased if it becomes a hindrance.} and is marked
%   with pseudo-pandoc delimiters. Everything in curly \texttt{\{\}}
%   brackets follows pandoc source code highlighting conventions.
%   \texttt{jodliterate}'s header contains \LaTeX~code that is marked like
%   this.

% \begin{verbatim}
% NB.<<~~~~ { .latex numberLines startFrom="1" }
% NB.>>~~~~ 
% \end{verbatim}
\end{enumerate}

\subsection{Running \texttt{jodliterate}}

\texttt{jodliterate} runs on Windows, Linux and Mac versions of
J.\footnote{\texttt{jodliterate} will run on J 8.01 and beyond.}

To use \texttt{jodliterate} you must:

\begin{enumerate}[1.]
\item
  Install JOD.
  \href{https://code.jsoftware.com/wiki/Addons/general/jod}{JOD} is a J
  addon. It is usually installed with
  \href{https://code.jsoftware.com/wiki/Pacman}{pacman}.
\item
  Install JODSOURCE.
  \href{https://code.jsoftware.com/wiki/Addons/general/jodsource}{JODSOURCE}
  is also a J addon. Install it with \texttt{pacman}. It contains the JOD
  dictionaries required to build JOD. JODSOURCE also contains
  \texttt{jodliterate.}
\item
  Install a current version of \href{https://pandoc.org/}{pandoc}.
  \texttt{pandoc} version \texttt{2.9.1.1} and beyond 
  supports J syntax highlighting. Prior versions required modification. The
  following blog posts provide more information:
      \begin{enumerate}
	  \item \href{https://analyzethedatanotthedrivel.org/2020/02/19/more-j-pandoc-syntax-highlighting/}{More J Pandoc Syntax HighLighting}
	  \item \href{https://analyzethedatanotthedrivel.org/2012/09/20/pandoc-based-j-syntax-highlighting/}{Pandoc based J Syntax Highlighting}
	  \item \href{https://analyzethedatanotthedrivel.org/2012/10/01/semi-literate-jod/}{Semi-Literate JOD}.
	  \end{enumerate}
\item
  Build \texttt{jodliterate.} \texttt{jodliterate} is a J group. JOD
  makes J scripts from groups. To make \texttt{jodliterate} do:

\begin{verbatim}
require 'general/jod'
od ;:'joddev jod utils' [ 3 od ''
mls 'jodliterate'
\end{verbatim}
\item
  Open the JOD dictionaries containing the group you want to document.

\begin{verbatim}
od ;:'joddev jod utils' [ 3 od ''
\end{verbatim}
\item
  Load \texttt{jodliterate.} After making \texttt{jodliterate} it can be
  loaded like any J script.

\begin{verbatim}
load 'jodliterate'
\end{verbatim}
\item
  Set a working directory. \texttt{jodliterate} generates
  \LaTeX~files. All such files are written to the directory specified by
  \texttt{setjodliterate.}

\begin{verbatim}
setjodliterate ''  NB. current JOD put dictionary document directory

setjodliterate 'c:\temp'          NB. windows
setjodliterate '/home/john/temp'  NB. linux
\end{verbatim}
\item
  Run \texttt{grplit} on the group you want to document. \texttt{grplit}
  generates \LaTeX~files in the working directory. The root \LaTeX~file
  is given the group name.

\begin{verbatim}
grplit 'jodliterate' NB. makes jodliterate.tex 
\end{verbatim}
\item
  Use \LaTeX~to compile the files generated by \texttt{grplit.} 

  A batch \texttt{bat} script is written to the \texttt{setjodliterate}
  directory that shows the sequence of \LaTeX\ compilation commands.
  
  Files
  may be processed with \texttt{pdflatex}, \texttt{xelatex} or
  \texttt{lualatex}. Your choice will be dictated by the presence of
  Unicode characters. For more details see the preamble file
  \texttt{JODLiteratePreamble.tex} in the working directory.
\end{enumerate}

There are some \LaTeX~settings you may want to adjust.

\begin{enumerate}[1.]
\item
  Edit \texttt{JODLiteratePreamble.tex} to change the number of index
  columns. The default is four but this may result in long names running
  together. To decrease or increase index columns change:

\begin{verbatim}
\begin{multicols}{4}[\section*{\indexname}]
\end{verbatim}
\item
  \texttt{jodliterate} wraps long source code lines. Wrapping is
  controlled by \texttt{WRAPLIMIT}. The default is 110 characters.
  Change this value if you change font or page size. Wrapped lines are
  preceeded with the string \texttt{WRAPLEAD} with default
  \AlertTok{\texttt{>..>}}.
\end{enumerate}
