% this jodliterate overview
\section{\texttt{jodliterate} Overview}

\texttt{jodliterate} is a J utility script that generates
\emph{literate} documents directly from JOD dictionary groups.

\subsection{\texttt{jodliterate} Interface}

\begin{Shaded}
\begin{Highlighting}[]
\hyperlink{grplit}{\NormalTok{grplit         }}\CommentTok{make latex for group (y)}
\hyperlink{setjodliterate}{\NormalTok{setjodliterate }}\CommentTok{prepare for processing}
\hyperlink{ifacesection}{\NormalTok{ifacesection   }}\CommentTok{interface section summary string}
\end{Highlighting}
\end{Shaded}


\subsection{\texttt{jodliterate} and JOD}

\texttt{jodliterate} makes some assumptions about the J code and
document text it processes. It assumes:

\begin{enumerate}[1.]
\item
  All source code is stored in JOD dictionaries.
\item
  JOD group documents \texttt{2 9 disp 'groupname'} are
  \href{http://johnmacfarlane.net/pandoc/README.html\#pandocs-markdown}{pandoc
  compatible markdown} text fragments. \emph{This is the most serious
  limitation of \texttt{jodliterate}.} Most of my group documents are
  not markdown. I am slowly converting them. If you compose group
  documents use pandoc markdown and \texttt{jodliterate} will be a lot
  more useful.
\item
  Any group class\footnote{A \emph{group class} is a JOD group with a
    declared list of interface and root words:
    \texttt{IFACEWORDS}\emph{groupname} and
    \texttt{ROOTWORDS}\emph{groupname}. Root words allow
    \texttt{allrefs} to find all the words that belong to the group.
    Class groups typically load into locales. \texttt{jodliterate.ijs}
    is an example of a class group.} interface is a simple J verbatim
  block that is marked with a \LaTeX~section command. Examine this
  group's document \texttt{2 9 disp 'jodliterate'} for an example.

\begin{verbatim}
% latex section 
\subsection{\texttt{groupname} Interface}

~~~~ { .j}
grplit          NB. make latex for group (y)
setjodliterate  NB. prepare for processing
~~~~~  
\end{verbatim}
\item
  Non-J code is confined to the JOD group header\footnote{This
    limitation may be eased if it becomes a hindrance.} and is marked
  with pseudo-pandoc delimiters. Everything in curly \texttt{\{\}}
  brackets follows pandoc source code highlighting conventions.
  \texttt{jodliterate}'s header contains \LaTeX~code that is marked like
  this.

\begin{verbatim}
NB.<<~~~~ { .latex numberLines startFrom="1" }
NB.>>~~~~ 
\end{verbatim}
\end{enumerate}

\subsection{Running \texttt{jodliterate}}

\texttt{jodliterate} runs on Windows, Linux and Mac versions of
J.\footnote{\texttt{jodliterate} will run on Windows versions of J 6.02
  and beyond. The Linux and Mac versions of \texttt{jodliterate} require
  J 7.01 and beyond.}

To use \texttt{jodliterate} you must:

\begin{enumerate}[1.]
\item
  Install JOD.
  \href{http://www.jsoftware.com/jwiki/Addons/general/jod}{JOD} is a J
  addon. It is usually installed with
  \href{http://www.jsoftware.com/jwiki/JAL}{JAL}.
\item
  Install JODSOURCE.
  \href{http://www.jsoftware.com/jwiki/Addons/general/jodsource}{JODSOURCE}
  is also a J addon. Install it with \texttt{pacman}.\footnote{On 64 bit versions of
    J the standard \texttt{pacman} install of JODSOURCE may fail. See
    \href{http://bakerjd99.wordpress.com/the-jod-page/\#comment-210}{The
    Jod Page} for an alternate install method.} It contains the JOD
  dictionaries required to build JOD. JODSOURCE also contains
  \texttt{jodliterate.}
\item
  Setup a version of \href{http://johnmacfarlane.net/pandoc/}{pandoc}
  that supports J syntax highlighting. See the blog post
  \href{http://bakerjd99.wordpress.com/2012/09/20/pandoc-based-j-syntax-highlighting/}{Pandoc
  based J Syntax Highlighting} for details.
\item
  Build \texttt{jodliterate.} \texttt{jodliterate} is a J group. JOD
  makes J scripts from groups. To make \texttt{jodliterate} do:

\begin{verbatim}
require 'general/jod'
od ;:'joddev jod utils' [ 3 od ''
mls 'jodliterate'
\end{verbatim}
\item
  Open the JOD dictionaries containing the group you want to document.

\begin{verbatim}
od ;:'joddev jod utils' [ 3 od ''
\end{verbatim}
\item
  Load \texttt{jodliterate.} After making \texttt{jodliterate} it can be
  loaded like any J script.

\begin{verbatim}
load 'jodliterate'
\end{verbatim}
\item
  Set a working directory. \texttt{jodliterate} generates markdown and
  \LaTeX~files. All such files are written to the directory specified by
  \texttt{setjodliterate.}

\begin{verbatim}
setjodliterate 'c:\temp'          NB. windows
setjodliterate '/home/john/temp'  NB. linux
\end{verbatim}
\item
  Run \texttt{grplit} on the group you want to document. \texttt{grplit}
  generates \LaTeX~files in the working directory. The root \LaTeX~file
  is given the group name.

\begin{verbatim}
grplit 'jodliterate' NB. makes jodliterate.tex 
\end{verbatim}
\item
  Use \LaTeX~to compile the files generated by \texttt{grplit.} Files
  may be processed with \texttt{pdflatex}, \texttt{xelatex} or
  \texttt{lualatex}. Your choice will be dictated by the presence of
  Unicode characters. For more details see the preamble file
  \texttt{JODLiteratePreamble.tex} in the working directory.
\end{enumerate}

There are some \LaTeX~settings you may want to adjust.

\begin{enumerate}[1.]
\item
  Edit \texttt{JODLiteratePreamble.tex} to change the number of index
  columns. The default is four but this may result in long names running
  together. To decrease or increase index columns change:

\begin{verbatim}
\begin{multicols}{4}[\section*{\indexname}]
\end{verbatim}
\item
  \texttt{jodliterate} wraps long source code lines. Wrapping is
  controlled by \texttt{WRAPLIMIT}. The default is 110 characters.
  Change this value if you change font or page size. Wrapped lines are
  preceeded with the string \texttt{WRAPLEAD} with default
  \AlertTok{\texttt{>...>}}.
\end{enumerate}
