% this jodliterate overview
\section{\texttt{sunmoon} Overview}

\texttt{sunmoon} is a collection of basic astronomical algorithms The
key verbs are \texttt{moons}, \texttt{sunriseset0} and
\texttt{sunriseset1.}\\
All of these verbs were derived from BASIC programs published in
\emph{Sky \& Telescope} magazine in the 1990's. The rest of the verbs in
\texttt{sunmoon} are mostly date and trigonometric utilities.

\subsection{\texttt{sunmoon} Interface}

\begin{Shaded}
\begin{Highlighting}[]
\hyperlink{calmoons}{\NormalTok{calmoons    }}\CommentTok{calendar dates of new and full moons}
\hyperlink{moons}{\NormalTok{moons       }}\CommentTok{times of new and full moons for n calendar years}
\hyperlink{sunriseset0}{\NormalTok{sunriseset0 }}\CommentTok{computes sun rise and set times - see group documentation}
\hyperlink{sunriseset1}{\NormalTok{sunriseset1 }}\CommentTok{computes sun rise and set times - see group documentation}
\end{Highlighting}
\end{Shaded}


\subsection{\textbf\texttt{sunriseset0} \textsl{v--} sunrise and sunset times}

This verb has been adapted from a BASIC program submitted by Robin G.
Stuart \emph{Sky \& Telescope's} shortest sunrise/set program contest.
Winning entries were listed in the March 1995 Astronomical Computing
column.

The J version of this algorithm has been vectorized. It can compute any
number of sunrise and sunset times in one call.

The \texttt{(y)} argument is a \texttt{5*n} floating point table where:

\begin{verbatim}
0{ is latitude in degrees with northern latitudes positive.
1{ is longitude in degrees with western longitudes negative.
2{ is western time zones expressed as positive whole hours.
3{ is the month number.
4{ is the day number.
\end{verbatim}

The result is a numeric table with four rows. To handle the cases when
the sun never rises or sets the first two elements of the corresponding
result columns are:

\begin{verbatim}
0{ is NORISESET an invalid hour indicating no rise or set
1{ is 0 when the sun never rises
1{ is 1 when the sun never sets
\end{verbatim}

Warning: this algorithm breaks for latitudes close to the South pole.

The original BASIC code has been slightly modified to use control
structures in place of GOTO's and line numbers.

Adapted from:

\begin{Shaded}
\begin{Highlighting}[numbers=left,,]
\NormalTok{/* Sunrise/set by R. G. Stuart,  Mexico City, Mexico */}
\NormalTok{PI = }\DecValTok{3.14159265}\NormalTok{\#: DR = PI / }\DecValTok{180}\NormalTok{: RD = }\DecValTok{1}\NormalTok{ / DR}
\FunctionTok{INPUT} \StringTok{"Lat, Long (deg)"}\CommentTok{; B5, L5}
\FunctionTok{INPUT} \StringTok{"Time zone (hrs)"}\CommentTok{; H}
\NormalTok{B5 = DR * B5}
\FunctionTok{INPUT} \StringTok{"Month, day"}\CommentTok{; M, D}
\NormalTok{N = }\FunctionTok{INT}\NormalTok{(}\DecValTok{275}\NormalTok{ * M / }\DecValTok{9}\NormalTok{) {-} }\DecValTok{2}\NormalTok{ * }\FunctionTok{INT}\NormalTok{((M + }\DecValTok{9}\NormalTok{) / }\DecValTok{12}\NormalTok{) + D {-} }\DecValTok{30}
\NormalTok{L0 = }\DecValTok{4.8771}\NormalTok{ + .}\DecValTok{0172}\NormalTok{ * (N + .}\DecValTok{5}\NormalTok{ {-} L5 / }\DecValTok{360}\NormalTok{)}
\NormalTok{C = .}\DecValTok{03342}\NormalTok{ * }\FunctionTok{SIN}\NormalTok{(L0 + }\DecValTok{1.345}\NormalTok{)}
\NormalTok{C2 = RD * (ATN(}\FunctionTok{TAN}\NormalTok{(L0 + C)) {-} ATN(.}\DecValTok{9175}\NormalTok{ * }\FunctionTok{TAN}\NormalTok{(L0 + C)) {-} C)}
\NormalTok{SD = .}\DecValTok{3978}\NormalTok{ * }\FunctionTok{SIN}\NormalTok{(L0 + C): CD = }\FunctionTok{SQR}\NormalTok{(}\DecValTok{1}\NormalTok{ {-} SD * SD)}
\NormalTok{SC = (SD * }\FunctionTok{SIN}\NormalTok{(B5) + .}\DecValTok{0145}\NormalTok{) / (}\FunctionTok{COS}\NormalTok{(B5) * CD)}
\KeywordTok{IF }\FunctionTok{ABS}\NormalTok{(SC) <= }\DecValTok{1}\NormalTok{ THEN}
\NormalTok{  C3 = RD * ATN(SC / }\FunctionTok{SQR}\NormalTok{(}\DecValTok{1}\NormalTok{ {-} SC * SC))}
\NormalTok{  R1 = }\DecValTok{6}\NormalTok{ {-} H {-} (L5 + C2 + C3) / }\DecValTok{15}
\NormalTok{  HR = }\FunctionTok{INT}\NormalTok{(R1): MR = }\FunctionTok{INT}\NormalTok{((R1 {-} HR) * }\DecValTok{60}\NormalTok{)}
  \FunctionTok{PRINT}\NormalTok{ USING }\StringTok{"Sunrise at \#\#:\#\#"}\CommentTok{; HR; MR}
\NormalTok{  S1 = }\DecValTok{18}\NormalTok{ {-} H {-} (L5 + C2 {-} C3) / }\DecValTok{15}
\NormalTok{  HS = }\FunctionTok{INT}\NormalTok{(S1): MS = }\FunctionTok{INT}\NormalTok{((S1 {-} HS) * }\DecValTok{60}\NormalTok{)}
  \FunctionTok{PRINT}\NormalTok{ USING }\StringTok{"Sunset at \#\#:\#\#"}\CommentTok{; HS; MS}
\KeywordTok{ELSEIF}\NormalTok{ SC > }\DecValTok{1}\NormalTok{ THEN}
  \FunctionTok{PRINT} \StringTok{"Sun up all day"}
\KeywordTok{ELSEIF}\NormalTok{ SC < {-}}\DecValTok{1}\NormalTok{ THEN}
  \FunctionTok{PRINT} \StringTok{"Sun down all day"}
\KeywordTok{END} \KeywordTok{IF}
\KeywordTok{END}
\end{Highlighting}
\end{Shaded}

\begin{Shaded}
\begin{Highlighting}[]
\NormalTok{monad}\KeywordTok{:}\NormalTok{ ntRiseset }\KeywordTok{=.}\NormalTok{ sunriseset0 flBLHMD}

  \CommentTok{NB. rise and set times at Dog Lake today (daylight savings)}
\NormalTok{  td}\KeywordTok{=.} \RegionMarkerTok{(}\DecValTok{44} \KeywordTok{+} \DecValTok{19}\KeywordTok{\%}\DecValTok{60}\RegionMarkerTok{)}\KeywordTok{,}\RegionMarkerTok{(}\KeywordTok{{-}} \DecValTok{76} \KeywordTok{+} \DecValTok{21}\KeywordTok{\%}\DecValTok{60}\RegionMarkerTok{)}\KeywordTok{,} \DecValTok{4} \KeywordTok{,} \KeywordTok{\}.}\NormalTok{ today }\DecValTok{0}
\NormalTok{  sunriseset0 td}

  \CommentTok{NB. rise and set times on June 30 on Greenwich meridian}
\NormalTok{  t0}\KeywordTok{=.}   \DecValTok{0} \DecValTok{0} \DecValTok{0} \DecValTok{6} \DecValTok{30}   \CommentTok{NB. equator}
\NormalTok{  t1}\KeywordTok{=.}  \DecValTok{49} \DecValTok{0} \DecValTok{0} \DecValTok{6} \DecValTok{30}   \CommentTok{NB. north {-} lat of western US/Canada border}
\NormalTok{  t2}\KeywordTok{=.} \DecValTok{\_47} \DecValTok{0} \DecValTok{0} \DecValTok{6} \DecValTok{30}   \CommentTok{NB. south {-} southern Chile and Argentina}
\NormalTok{  t3}\KeywordTok{=.}  \DecValTok{75} \DecValTok{0} \DecValTok{0} \DecValTok{6} \DecValTok{30}   \CommentTok{NB. far north (sun always up)}
\NormalTok{  t4}\KeywordTok{=.} \DecValTok{\_75} \DecValTok{0} \DecValTok{0} \DecValTok{6} \DecValTok{30}   \CommentTok{NB. far south (sun always down)}

\NormalTok{  sunriseset0 t0}

\NormalTok{  sunriseset0 t0 }\KeywordTok{,}\NormalTok{ t1 }\KeywordTok{,}\NormalTok{ t2 }\KeywordTok{,}\NormalTok{ t3 }\KeywordTok{,:}\NormalTok{ t4}

  \CommentTok{NB. times on equator for March 21 for all 1 hour time zones}
\NormalTok{  sunriseset0  }\DecValTok{0} \DecValTok{0} \KeywordTok{,"}\DecValTok{1} \RegionMarkerTok{(}\KeywordTok{,.i.} \DecValTok{24}\RegionMarkerTok{)} \KeywordTok{,"}\DecValTok{1} \KeywordTok{]} \DecValTok{3} \DecValTok{21}

  \CommentTok{NB. times for calendar year 1995 on the Greenwich meridian}
\NormalTok{  md95}\KeywordTok{=.}  \DecValTok{47} \DecValTok{0} \DecValTok{0} \KeywordTok{,"}\DecValTok{1} \KeywordTok{\}."}\DecValTok{1}\NormalTok{ yeardates }\DecValTok{1995}
\NormalTok{  rs095}\KeywordTok{=.}\NormalTok{ sunriseset0 md95}
\end{Highlighting}
\end{Shaded}

\subsection{\textbf\texttt{sunriseset1} \textsl{v--} sunrise and sunset times}

This verb has been adapted from a BASIC program submitted by James
Brimhall to \emph{Sky \& Telescope's} ``shortest sunrise/set program''
contest. Winning entries were listed in the March 1995 Astronomical
Computing column.

The \texttt{(y)} argument of \texttt{sunriseset1} is a \texttt{6*n}
floating point table where:

\begin{verbatim}
0{ is latitude in degrees with northern latitudes positive.
1{ is longitude in degrees with western longitudes negative.
2{ is western time zones expressed as positive whole hours.
3{ is the month number.
4{ is the day number.
5{ is the year.
\end{verbatim}

The result is a numeric table with four rows. To handle the cases when
the sun never rises or sets the first two elements of the corresponding
result columns n are:

\begin{verbatim}
0{ is NORISESET an invalid hour that indicates no rise or set.
1{ is 0 when the sun never rises.
1{ is 1 when the sun never sets.
\end{verbatim}

Adapted from:

\begin{Shaded}
\begin{Highlighting}[numbers=left,,]
\NormalTok{/* Sunrise/set by James Brimhall, St. Albans, WV */}
\NormalTok{PI = }\DecValTok{3.1415927}\NormalTok{\#: DR = PI / }\DecValTok{180}\NormalTok{: DD = }\DecValTok{360}\NormalTok{ / }\DecValTok{365.25636}\NormalTok{\#: }\KeywordTok{DIM}\NormalTok{ D(}\DecValTok{20}\NormalTok{)}
\KeywordTok{DATA} \DecValTok{0}\NormalTok{,}\DecValTok{31}\NormalTok{,}\DecValTok{59}\NormalTok{,}\DecValTok{90}\NormalTok{,}\DecValTok{120}\NormalTok{,}\DecValTok{151}\NormalTok{,}\DecValTok{181}\NormalTok{,}\DecValTok{212}\NormalTok{,}\DecValTok{243}\NormalTok{,}\DecValTok{273}\NormalTok{,}\DecValTok{304}\NormalTok{,}\DecValTok{334}
\KeywordTok{FOR }\NormalTok{C = }\DecValTok{1} \KeywordTok{TO} \DecValTok{12}\NormalTok{: }\KeywordTok{READ}\NormalTok{ D(C): }\KeywordTok{NEXT }\NormalTok{C}
\FunctionTok{INPUT} \StringTok{"Lat, long"}\CommentTok{; LA, LO}
\FunctionTok{INPUT} \StringTok{"Month, day, year"}\CommentTok{; M, D, Y}
\KeywordTok{IF }\NormalTok{Y / }\DecValTok{4}\NormalTok{ = }\FunctionTok{INT}\NormalTok{(Y / }\DecValTok{4}\NormalTok{) THEN }\KeywordTok{FOR }\NormalTok{C = }\DecValTok{3} \KeywordTok{TO} \DecValTok{12}\NormalTok{: D(C) = D(C) + }\DecValTok{1}\NormalTok{: }\KeywordTok{NEXT }\NormalTok{C}
\NormalTok{DY = D(M) + D: DY = DY {-} LO / }\DecValTok{360}\NormalTok{: DY = DY + .}\DecValTok{5}
\NormalTok{TH = }\DecValTok{9.357001}\NormalTok{ + DD * DY + }\DecValTok{1.914}\NormalTok{ * }\FunctionTok{SIN}\NormalTok{(DR * (DD * DY {-} }\DecValTok{3.97}\NormalTok{))}
\NormalTok{C3 = .}\DecValTok{3978}\NormalTok{ * }\FunctionTok{COS}\NormalTok{(DR * TH)}
\NormalTok{DC = {-}}\DecValTok{1}\NormalTok{ / DR * ATN(C3 / (}\FunctionTok{SQR}\NormalTok{(}\DecValTok{1}\NormalTok{ {-} C3 \^{} }\DecValTok{2}\NormalTok{)))}
\KeywordTok{IF }\NormalTok{LA < }\DecValTok{0}\NormalTok{ THEN A1 = }\DecValTok{90}\NormalTok{ + LA {-} DC }\KeywordTok{ELSE}\NormalTok{ A1 = }\DecValTok{90}\NormalTok{ {-} LA + DC: }\FunctionTok{PRINT}
\KeywordTok{IF }\NormalTok{A1 < {-}}\DecValTok{50}\NormalTok{ / }\DecValTok{60}\NormalTok{ THEN }\FunctionTok{PRINT} \StringTok{"Sun never rises"}\NormalTok{: }\KeywordTok{GOTO} \DecValTok{200}
\KeywordTok{IF }\NormalTok{LA < }\DecValTok{0}\NormalTok{ THEN A2 = {-}}\DecValTok{90}\NormalTok{ {-} LA {-} DC }\KeywordTok{ELSE}\NormalTok{ A2 = LA {-} }\DecValTok{90}\NormalTok{ + DC}
\KeywordTok{IF }\NormalTok{A2 >= {-}}\DecValTok{50}\NormalTok{ / }\DecValTok{60}\NormalTok{ THEN }\FunctionTok{PRINT} \StringTok{"Sun never sets"}\NormalTok{: }\KeywordTok{GOTO} \DecValTok{200}
\NormalTok{C1 = (}\FunctionTok{SIN}\NormalTok{({-}DR * }\DecValTok{50}\NormalTok{ / }\DecValTok{60}\NormalTok{) {-} }\FunctionTok{SIN}\NormalTok{(DR * DC) * }\FunctionTok{SIN}\NormalTok{(DR * LA)) / (}\FunctionTok{COS}\NormalTok{(DR * DC) * }\FunctionTok{COS}\NormalTok{(DR * LA))}
\NormalTok{T2 = (}\DecValTok{1}\NormalTok{ / DR) * ATN(}\FunctionTok{SQR}\NormalTok{(}\DecValTok{1}\NormalTok{ {-} C1 \^{} }\DecValTok{2}\NormalTok{) / C1): }\KeywordTok{IF }\NormalTok{C1 >= }\DecValTok{0}\NormalTok{ THEN T1 = }\DecValTok{360}\NormalTok{ {-} T2}
\KeywordTok{IF }\NormalTok{C1 < }\DecValTok{0}\NormalTok{ THEN T2 = }\DecValTok{180}\NormalTok{ + T2: T1 = }\DecValTok{360}\NormalTok{ {-} T2}
\NormalTok{TR = T1 / }\DecValTok{15}\NormalTok{ {-} }\DecValTok{12}\NormalTok{: TS = T2 / }\DecValTok{15}
\NormalTok{ET = {-}.}\DecValTok{1276}\NormalTok{ * }\FunctionTok{SIN}\NormalTok{(DR * (DD * DY {-} }\DecValTok{3.97}\NormalTok{)) {-} .}\DecValTok{1511}\NormalTok{ * }\FunctionTok{SIN}\NormalTok{(DR * (}\DecValTok{2}\NormalTok{ * DD * DY + }\DecValTok{17.86}\NormalTok{))}
\NormalTok{TR = TR {-} ET: TS = TS {-} ET}
\FunctionTok{INPUT} \StringTok{"Time zone (h)"}\CommentTok{; TZ}
\NormalTok{TC = {-}TZ {-} LO / }\DecValTok{15}\NormalTok{: R = TR + TC: S = TS + TC}
\NormalTok{C2 = (}\FunctionTok{SIN}\NormalTok{(DR * DC) {-} }\FunctionTok{SIN}\NormalTok{(DR * LA) * }\FunctionTok{SIN}\NormalTok{({-}DR * }\DecValTok{50}\NormalTok{ / }\DecValTok{60}\NormalTok{)) / (}\FunctionTok{COS}\NormalTok{(DR * LA) {-} }\FunctionTok{SIN}\NormalTok{(DR * LA) * }\FunctionTok{SIN}\NormalTok{({-}DR * }\DecValTok{50}\NormalTok{ / }\DecValTok{60}\NormalTok{))}
\NormalTok{Z1 = (}\DecValTok{1}\NormalTok{ / DR) * ATN(}\FunctionTok{SQR}\NormalTok{(}\DecValTok{1}\NormalTok{ {-} C2 \^{} }\DecValTok{2}\NormalTok{) / C2): }\KeywordTok{IF }\NormalTok{C2 >= }\DecValTok{0}\NormalTok{ THEN Z2 = }\DecValTok{360}\NormalTok{ {-} Z1}
\KeywordTok{IF }\NormalTok{C2 < }\DecValTok{0}\NormalTok{ THEN Z1 = }\DecValTok{180}\NormalTok{ + Z1: Z2 = }\DecValTok{360}\NormalTok{ {-} Z1}
\FunctionTok{PRINT} \StringTok{"Sunrise "}\CommentTok{; INT(R); "h"; INT(600 * (R {-} INT(R))) / 10;}
\FunctionTok{PRINT} \StringTok{"m a.m., azimuth "}\CommentTok{; INT(10 * Z1) / 10}
\FunctionTok{PRINT} \StringTok{"Sunset "}\CommentTok{; INT(S); "h"; INT(600 * (S {-} INT(S))) / 10;}
\FunctionTok{PRINT} \StringTok{"m p.m., azimuth "}\CommentTok{; INT(10 * Z2) / 10}
\DecValTok{200} \KeywordTok{END}
\end{Highlighting}
\end{Shaded}

\begin{Shaded}
\begin{Highlighting}[]
\NormalTok{monad}\KeywordTok{:}\NormalTok{  ntRiseset }\KeywordTok{=.}\NormalTok{ sunriseset1 flBLHMDY}

  \CommentTok{NB. rise and set times at Dog Lake today (daylight savings)}
\NormalTok{  td}\KeywordTok{=.} \RegionMarkerTok{(}\DecValTok{44} \KeywordTok{+} \DecValTok{19}\KeywordTok{\%}\DecValTok{60}\RegionMarkerTok{)}\KeywordTok{,}\RegionMarkerTok{(}\KeywordTok{{-}} \DecValTok{76} \KeywordTok{+} \DecValTok{21}\KeywordTok{\%}\DecValTok{60}\RegionMarkerTok{)}\KeywordTok{,} \DecValTok{4} \KeywordTok{,} \DecValTok{1} \KeywordTok{|.}\NormalTok{ today }\DecValTok{0}
\NormalTok{  sunriseset1 td}

  \CommentTok{NB. rise and set times on June 30 1995 on Greenwich meridian}
\NormalTok{  t0}\KeywordTok{=.}   \DecValTok{0} \DecValTok{0} \DecValTok{0} \DecValTok{6} \DecValTok{30} \DecValTok{1995}  \CommentTok{NB. equator}
\NormalTok{  t1}\KeywordTok{=.}  \DecValTok{49} \DecValTok{0} \DecValTok{0} \DecValTok{6} \DecValTok{30} \DecValTok{1995}  \CommentTok{NB. north {-} lat of western US/Canada border}
\NormalTok{  t2}\KeywordTok{=.} \DecValTok{\_47} \DecValTok{0} \DecValTok{0} \DecValTok{6} \DecValTok{30} \DecValTok{1995}  \CommentTok{NB. south {-} southern Chile and Argentina}
\NormalTok{  t3}\KeywordTok{=.}  \DecValTok{75} \DecValTok{0} \DecValTok{0} \DecValTok{6} \DecValTok{30} \DecValTok{1995}  \CommentTok{NB. far north (sun always up)}
\NormalTok{  t4}\KeywordTok{=.} \DecValTok{\_75} \DecValTok{0} \DecValTok{0} \DecValTok{6} \DecValTok{30} \DecValTok{1995}  \CommentTok{NB. far south (sun always down)}

\NormalTok{  sunriseset1 t0}

\NormalTok{  sunriseset1 t0 }\KeywordTok{,}\NormalTok{ t1 }\KeywordTok{,}\NormalTok{ t2 }\KeywordTok{,}\NormalTok{ t3 }\KeywordTok{,:}\NormalTok{ t4}

  \CommentTok{NB. compare algorithms}
\NormalTok{  sun1}\KeywordTok{=.}\NormalTok{ sunriseset1 t0 }\KeywordTok{,}\NormalTok{ t1 }\KeywordTok{,}\NormalTok{ t2 }\KeywordTok{,}\NormalTok{ t3 }\KeywordTok{,:}\NormalTok{ t4}
\NormalTok{  sun0}\KeywordTok{=.}\NormalTok{ sunriseset0 }\KeywordTok{\}:"}\DecValTok{1}\NormalTok{ t0 }\KeywordTok{,}\NormalTok{ t1 }\KeywordTok{,}\NormalTok{ t2 }\KeywordTok{,}\NormalTok{ t3 }\KeywordTok{,:}\NormalTok{ t4}
\NormalTok{  sun1 }\KeywordTok{{-}}\NormalTok{ sun0}

  \CommentTok{NB. times on equator for March 21 1995 for all 1 hour time zones}
\NormalTok{  sunriseset1 }\DecValTok{0} \DecValTok{0} \KeywordTok{,"}\DecValTok{1} \RegionMarkerTok{(}\KeywordTok{,.i.} \DecValTok{24}\RegionMarkerTok{)} \KeywordTok{,"}\DecValTok{1} \KeywordTok{]} \DecValTok{3} \DecValTok{21} \DecValTok{1995}

  \CommentTok{NB. times for calendar year 1995 on the Greenwich meridian}
\NormalTok{  mdy95}\KeywordTok{=.} \DecValTok{47} \DecValTok{0} \DecValTok{0} \KeywordTok{,"}\DecValTok{1} \KeywordTok{]} \DecValTok{1} \KeywordTok{|."}\DecValTok{1}\NormalTok{ yeardates }\DecValTok{1995}
\NormalTok{  rs195}\KeywordTok{=.}\NormalTok{ sunriseset1 mdy95}
\end{Highlighting}
\end{Shaded}
