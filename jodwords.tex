\section{JOD Interface Words}\label{ss:jodwords}

This section describes all JOD user interface words in alphabetical
order.  Each  word description consists of a short explanation followed
by examples.  Examples show J inputs; outputs
are omitted.  Word arguments are summarized with a form of \emph{Hungarian}
notation: see appendix~\ref{ap:jodhung} on
page~\pageref{ap:jodhung}.

When JOD loads it creates a number of locale classes,  (see appendix~\ref{ap:classes}
on page~\pageref{ap:classes}), and defines an \texttt{\textbf{ijod}}\index{ijod interface} user 
interface locale: \texttt{\textbf{ijod}} 
is placed on the \texttt{\textbf{base}} locale's path.
JOD's \texttt{\textbf{ijod}} user interface consists of definitions like:  
\begin{lstlisting}[frame=single,framerule=0pt]
   get     NB. display get interface
get_1_ ::jodsf
    
   mls     NB. display mls interface
mls_8_ ::jodsf

   jodsf   NB. display error trap
0"_ ; 'JOD SYSTEM FAILURE: last J error -> '"_ , [: 13!:12 ''"_ [ ]   
\end{lstlisting}
Each interface word calls a corresponding object instance word.  The
default interface is an \emph{error trapping} interface.\footnote{Provisions
for defining a \emph{non-error-trapping} interface exist in JOD source code.}  If any JOD
word fails \texttt{jodsf} will catch the error and return JOD's standard \verb|(paRc;clMessage)|
result.

\subsection{\texttt{addgrp} --- add words/tests to group/suite}

\texttt{addgrp} adds words to a group\index{group!\texttt{addgrp}} and tests to a suite. 

\begin{wordhead}
\dyad & clGroup addgrp clName \argsep blclNames \\
      & (clSuite;iaObject) addgrp clName \argsep blclNames \\
\end{wordhead}
\begin{lstlisting}[frame=single,framerule=0pt]
  NB. add a word to a group
  'group' addgrp 'word' 

  NB. add many words to a group 
  'groupname' addgrp ;:'word names to group'

  NB. boxed (x) is used for suites - 3 denotes suite
  ('suitename';3) addgrp ;:'tests added to suite'
\end{lstlisting}

\subsection{\texttt{bget} --- backup get}\label{ss:bget}\index{backup!\texttt{bget}} 

  \hypertarget{il:bget}{\texttt{bget}} fetches objects from either the last dictionary \emph{binary} backup or particular backups.
  Backups are created by \hyperlink{il:packd}{\texttt{packd}}, see page~\pageref{ss:packd}.
  
  Fetched \emph{objects are not defined in locales} for the simple reason that
  backup fetches may return many versions of the same object.
  \texttt{bget} like \hyperlink{il:bnl}{\texttt{bnl}}, only searches put dictionary\index{put dictionary} backups 
  and unlike \hyperlink{il:get}{\texttt{get}}, (see page~\pageref{ss:get}), \texttt{bget} does 
  not fetch name classes, timestamps or byte sizes. 
  
  The \texttt{y} argument of \texttt{bget} is similar to the \texttt{y} argument of 
  \hyperlink{il:get}{\texttt{get}}. Both take
  list of names. \texttt{get} takes lists of J names that do not reference
  locales or objects and \texttt{bget} takes \texttt{get} names and
  \emph{backup names} where a backup name is a \texttt{get} name that may end with a \emph{backup suffix}.
  The following \texttt{bget} examples illustrate backup names.
   
    \begin{wordhead}
\monad & bget clBName \argsep blclBNames \\
\end{wordhead}
\begin{lstlisting}[frame=single,framerule=0pt]
  NB. get word from last backup
  bget 'oops'
  
  NB. fetch words from last backup
  bget ;: 'shawn of the dead'
  
  NB. fetch words from particular backups - backup names
  bget <;._1 ' us.12 poor.10 little.08 words.08 lastback'
  
  NB. fetch many versions of a word - backup names
  bget <;._1 ' me me.12 me.09 me.08 me.02'
  
  NB. all words from last backup
  bget }. bnl ''
  
  NB. all words from backup #55
  bget (}.bnl '.55') ,&.> <'.55'
\end{lstlisting}

\begin{wordhead}
\dyad & iaObject bget clBName \argsep blclBNames \argsep uuIgnore \\
      & (iaObject,iaOption) bget clBName \argsep blclNames \\
\end{wordhead}
\begin{lstlisting}[frame=single,framerule=0pt]
  5 bget ''     NB. dictionary document from last backup
  5 bget '.12'  NB. dictionary document from particular backup
  5 bget }. bnl '.'  NB. documents from all backups

  NB. fetch a suite header from a particular backup
  3 bget 'sweet.04'
  
  NB. fetch a particular suite test list 
  3 1 bget 'sweet.04'

  NB. three versions of a group's header - similar to (get) 
  NB. where (2 get 'group') returns the group header
  2 bget <;._1 ' gfoo.12 gfoo.05 gfoo.00'  
 
  NB. three versions of a group's word list - backup names
  2 1 bget <;._1 ' gfoo.12 gfoo.05 gfoo.00'  
  
  NB. fetch many versions of a macro - backup names
  4 bget <;._1 ' mymacro.13 mymacro.09 mymacro.07 mymacro.02'   
\end{lstlisting}

\subsection{\texttt{bnl} --- backup name lists}\label{ss:bnl}\index{backup!\texttt{bnl}}\index{search!names!\texttt{bnl}}

  \hypertarget{il:bnl}{\texttt{bnl}} searches\index{dictionary!search!\texttt{bnl}} \emph{put dictionary} backups, 
  see page~\pageref{xr:putdic} and appendix~\ref{ap:refpath} on page~\pageref{ap:refpath}, and returns backup name lists. 
  \texttt{bnl}'s arguments are similiar to \hyperlink{il:dnl}{\texttt{dnl}}, see page~\pageref{ss:dnl}.
  
  \begin{wordhead}
\monad & bnl zl \argsep clPstr \\
\end{wordhead}
\begin{lstlisting}[frame=single,framerule=0pt]
  NB. list all put dictionary backup numbers
  bnl '.'
  
  NB. words in last put dictionary backup
  bnl ''
  
  NB. all words in all backups
  bnl&> }. bnl '.'
 
  NB. words in last backup matching prefix 
  bnl 'prefix'
  
  NB. words in backup #42 matching prefix
  bnl 'prefix.42'
\end{lstlisting}

\begin{wordhead}
\dyad & iaObject bnl zl \argsep clPstr \\
      & (iaObject,iaOption) bnl zl \argsep clPstr \\
      & (iaObject,iaOption,iaQualifier) bnl zl  \\
      & (iaObject,iaOption,iaQualifier) bnl clPstr \\
\end{wordhead}
\begin{lstlisting}[frame=single,framerule=0pt]
  0 bnl ''     NB. all words in last backup (monad)
  0 bnl '.11'  NB. all words in backup #11
  
  1 bnl ''     NB. all tests in last backup
  1 bnl '.13'  NB. all tests in backup #13
  
  2 bnl ''     NB. all groups in last backup
  2 bnl '.7'   NB. all groups in backup #7

  NB. suites in backup #14 beginning with boo
  3 bnl 'boo.14'  
  
  NB. words in last backup containing the string str
  0 2 bnl 'str'
  
  NB. words in backup #6 containg the string ass
  0 2 bnl 'ass.6'  
  
  NB. words in all backups containing yada
  0 2&bnl&> (<'yada') ,&.> }. bnl '.'
  
  NB. word names in last backup ending with string str 
  0 3 bnl 'str'  

  NB. adverbs in backup #17 beginning with str
  0 1 1 bnl 'str.17'  
  
  NB. verb names in last backup containing str
  0 2 3 bnl 'str'  
  
  NB. nouns in backup #9 ending with str
  0 3 0 bnl 'str.9'
  
  NB. nouns in all backups ending with EX 
  0 3 0&bnl&> (<'EX') ,&.> }. bnl '.'  

  NB. J macro names beginning with jscript
  4 1 21 bnl 'jscript' 

  NB. MARKDOWN macro names in backup #18 containing mark
  4 2 27 bnl 'mark.18'   
  
  NB. TEXT macro names in backup #23 ending badly 
  4 3 25 bnl 'badly.23' 
  
  NB. put dictionary backup numbers and file timestamps
  14 bnl '.'
\end{lstlisting}
 

\subsection{\texttt{compj} --- compress J code}

  \texttt{compj} compresses\index{compression} J code by
  removing comments, white space and shortening safe local identifiers to single
  characters.\footnote{If more than one character is required to rename all
  local identifiers \texttt{compj} uses a letter prefixed high base numbering scheme. }
  Code compression is useful when preparing production scripts. The
  JOD system script:
  
\begin{verbatim}
  ~addons/general/jod/jod.ijs
\end{verbatim}

  is an example of a compressed J script. In its fully commented
  form this script is about 230 kilobytes when squeezed with
  \texttt{compj} it shrinks to about 84 kilobytes.
  \texttt{compj} does not compress words in JOD
  dictionaries it returns a compressed script result.
  
\begin{quotation}
 \noindent\textcolor{red}{Warning: to safely use
	  \texttt{compj} you must understand how to:
     \begin{enumerate} 
      \item mark ambiguous names.
      \item exclude words with local names that
        match quoted text. 
     \end{enumerate}
	  \emph{If you do not properly mark ambiguous names and exclude words
     with local names that match quoted text 
    \texttt{compj} will break your code!}}
\end{quotation}
  
  Prior to compressing a word apply \hyperlink{il:globs}{\texttt{globs}}, 
  see page~\pageref{ss:globs}, to expose any name problems.

  Ambiguous names in J are words created in object instances,
  temporary locale globals, names masked by indirect assignments\index{assignments!indirect} and objects
  created with execute. When you use ambiguous names augment your code with
  sufficient information to clearly resolve and cross reference
  \index{cross references!names} all names. 
  
  JOD provides two comment\index{scope tags} scope tags
   \textcolor{CodeComment}{\texttt{\textsl{(*)=.}}}\index{comment tag!\texttt{(*)=.}}and
  \textcolor{CodeComment}{\texttt{\textsl{(*)=:}}}\index{comment tag!\texttt{(*)=:}} to clarify ambiguous names.  Comment
  scope tags override J scope.  

\begin{enumerate}
	\item local tag \textcolor{CodeComment}{\texttt{\textsl{ NB. (*)=. local names declared after tag}}}
	\item global tag \textcolor{CodeComment}{\texttt{\textsl{ NB. (*)=: global names also declared}}}
\end{enumerate}
  
   The following examples show how to use these tags:

\begin{lstlisting}[frame=single,framerule=0pt]  
indirectassignments=: 4 : 0

NB. Indirect assignments ()=: create objects 
NB. that elude static cross referencing. 
NB. Declaring the names global and local
NB. makes it possible to cross reference 
NB. this verb with (globs)
globref=.  ;:'one two three'

NB. declared global (*)=: one two three
(globref)=: y     

NB. declared local (*)=. we are hidden locals
locref=.   ;:'we are hidden locals'
(locref)=. i. 4  

NB. without tags these names appear to
NB. be used out of nowhere 
one * two * three
we + are + hidden + locals
)
\end{lstlisting}

     
\begin{lstlisting}[frame=single,framerule=0pt]    
createobject=: 3 : 0

NB. Object initialization often creates
NB. global nouns that are not really globals.
NB. They only exist within the the scope of
NB. the object. Tags can over ride J's 
NB. global scope for cross referencing.

NB. create "globals" in an object instance
THIS=: STUFF=: IS=: INSIDE=: AN=: OBJECT=: 1

NB. over ride J's scope by declaring names local.
NB. !(*)=. THIS STUFF IS INSIDE AN OBJECT
1
)
\end{lstlisting}

 \texttt{compj} recognizes two additional comment tags \textcolor{CodeComment}{\texttt{\textsl{(-.)=:}}}
 and \textcolor{CodeComment}{\texttt{\textsl{(/:)=:}}}. The first limits compression to only white space removal
 and the second \emph{obfuscates} local names.
 
\begin{enumerate}
\item
  \texttt{compj} compression can be limited to \emph{white space removal} by
  marking explicit words with the comment tag \textcolor{CodeComment}{\texttt{\textsl{(-.)=:}}}.\index{comment tag!\texttt{(-.)=:}}

\begin{lstlisting}[frame=single,framerule=0pt]
donotsqueeze=: 4 : 0

NB. I have local names that occur in quoted strings.
NB. Compressing me will end civilization as we know it.
NB. Use the comment tag (-.)=: anywhere to save the day.

NB. do not compress (-.)=:
home=. 2 * y
heart=. x - 1
if. 0=|home do.
  'home is where the heart is'
else.
  'heart health helps'
end.
)
\end{lstlisting}

\item
\texttt{compj} will obfuscate local names when explicit words are marked with the comment tag
\textcolor{CodeComment}{\texttt{\textsl{(/:)=:}}}\index{comment tag!\texttt{(/:)=:}}. Name obfuscation
makes code difficult to read and should only be used when words are running general scripts with 
\verb|0!:k| foreigns. 
\textcolor{CodeComment}{\texttt{\textsl{(-.)=:}}} overrides \textcolor{CodeComment}{\texttt{\textsl{(/:)=:}}}.

\begin{lstlisting}[frame=single,framerule=0pt]
obfuscateme=: 3 : 0

NB. I run a script that may reassign my names.
NB. Obfuscation makes this unlikely.

NB. (/:)=: do not eschew obfuscation
script=. y

NB. running (script) may clash with local (script)
0!:000 y

'bytes run' , ":# script
)
\end{lstlisting}

\end{enumerate}

    More examples of the use of comment tags can be found in 
      JOD source code.  JOD source code is not 
     distributed with JOD.  You can get JOD source code by installing
     the  \href{http://www.jsoftware.com/jwiki/Addons/general/jodsource}{\texttt{jodsource}}
     addon or by downloading \href{http://bakerjd99.wordpress.com/the-jod-page/}{\texttt{jodijs.zip}} from %point directly to box file
     \href{http://bakerjd99.wordpress.com/the-jod-page/}{\emph{The JOD Page}} \cite{baker:jodpages}.  
     JOD source is distributed as JOD dictionary dump scripts and as fully commented
	 class files in \texttt{jodijs.zip}.

\begin{wordhead}
\monad & compj clName \argsep blclNames \\
\end{wordhead}
\begin{lstlisting}[frame=single,framerule=0pt]
  NB. compress a single word
  compj 'squeezeme'  
  
  NB. compress words beginning with fat
  compj }. dnl 'fat' 

  NB. compress all words in a group 
  'rc script'=. compj }. grp 'group'
\end{lstlisting}

\begin{wordhead}
\dyad & iaOption compj clName \argsep blclNames \\
\end{wordhead}
\begin{lstlisting}[frame=single,framerule=0pt]
  NB. remove comments preserving leading whitespace and
  NB. original identifiers - useful for reading code
  NB. without distracting comments
  >1{ 1 compj 'justthecode'   
\end{lstlisting}

\subsection{\texttt{del} --- delete objects}

\hypertarget{il:del}{\texttt{del}} deletes dictionary objects. 
If objects are on the search path but not
 in the put dictionary\index{put dictionary} nothing will be deleted and 
 the \emph{nonput dictionary} objects will be identified in an error message. 

Warning: \texttt{del} will remove objects that are in use without warning.
 This can lead to broken groups and suites.\footnote{In database
 terms \texttt{del} can violate \emph{referential integrity}. Early versions
 of JOD maintained referential integrity but this proved cumbersome and
 was dropped.} Deleting a word that belongs
 to a group breaks the group:  similarly for suites.
 An attempt to \hyperlink{il:get}{\texttt{get}} or \hyperlink{il:make}{\texttt{make}}
  a broken group or suite will result in an error.  You can recover from this
  error by deleting references, (see below), and regrouping.
  
\begin{wordhead}
\monad &  del clName \argsep blclNames \\
\end{wordhead}
\begin{lstlisting}[frame=single,framerule=0pt]
  NB. delete one word
  del 'word'  
  
  NB. delete many words                   
  del 'go';'ahead';'delete';'us' 
\end{lstlisting}

\begin{wordhead}
\dyad & iaObject del clName \argsep blclNames \\
\end{wordhead}
\begin{lstlisting}[frame=single,framerule=0pt]
  NB. delete a test
  1 del 'test'  
  
  NB. delete a group - words in the
  NB. group are not not deleted 
  2 del 'group' 

  NB. delete many groups
  2 del ;:'we are toast' 
  
  NB. delete suites and macros 
  3 del 'suite'           
  4 del 'macro'
  
  NB. delete many macros
  4 del 'macro';'byebye'  
  
  NB. delete word references  
  11 del ;:'remove our references'  
\end{lstlisting}

\subsection{\texttt{delgrp} ---  remove words/tests from group/suite}

\texttt{delgrp} removes words from a group\index{group!\texttt{delgrp}} and tests from a suite.  

Removing objects from groups and suites does not delete them. To delete objects use 
\hyperlink{il:del}{\texttt{del}}.

\begin{wordhead}
\dyad & clGroup delgrp clName \argsep blclNames \\
      & (clSuite;iaObject) delgrp clName \argsep blclNames \\
\end{wordhead}
\begin{lstlisting}[frame=single,framerule=0pt]
  NB. remove a word from a group
  'group' delgrp 'word' 

  NB. remove many words from a group 
  'groupname' delgrp ;:'word names to remove'

  NB. boxed (x) is used for suites - 3 denotes suite
  ('suitename';3) delgrp ;:'tests removed from suite'
\end{lstlisting}

 
\subsection{\texttt{did} --- dictionary identification}\label{ss:did}

\hypertarget{il:did}{\texttt{did}} identifies open dictionaries. 

\begin{wordhead}
\monad & did uuIgnore \\
\end{wordhead}
\begin{lstlisting}[frame=single,framerule=0pt]
  NB. lists open dictionaries in path order 
  did 0 
\end{lstlisting}

\begin{wordhead}
\dyad & uuIgnore did uuIgnore \\
\end{wordhead}
\begin{lstlisting}[frame=single,framerule=0pt]
  NB. open dictionaries and basic statistics 
  0 did 0  

  NB. handy idiom 
  did~ 0    
\end{lstlisting}

\subsection{\texttt{disp} --- display dictionary objects}\label{ss:disp} 

\hypertarget{il:disp}{\texttt{disp}} displays dictionary objects. \texttt{disp} returns a character 
list when successful and the standard boxed \verb|(paRc;clMessage)| when reporting errors. 

\begin{wordhead}
\monad & disp clName \argsep blclNames \\
\end{wordhead}
\begin{lstlisting}[frame=single,framerule=0pt]
  NB. display a word 
  disp 'word'     

  NB. display many words 
  disp ;:'go ahead show us' 
\end{lstlisting}

\begin{wordhead}
\dyad & iaObject disp clName \argsep blclNames \\
      & (iaObject,iaOption) disp clName \argsep blclNames \\
\end{wordhead}
\begin{lstlisting}[frame=single,framerule=0pt]
  NB. show a test 
  1 disp 'test'  
  
  NB. generate and display a group 
  2 disp 'group' 

  NB. display group header 
  2 1 disp 'group' 
  
  NB. display group documentation 
  2 9 disp 'group' 
  
  NB. display all short group explanations
  2 8 disp }. grp ''    

  NB. generate and display a suite 
  3 disp 'suite' 

  NB. display suite header 
  3 1 disp 'suite'  
  
  NB. display suite documentation 
  3 9 disp 'suite'     

  NB. display one macro
  4 disp 'macro' 
  
  NB. display many macros           
  4 disp 'macro';'byebye' 
  
  NB. display put dictionary documentation
  5 disp ''      
\end{lstlisting}
  
\subsection{\texttt{dnl} --- dictionary name lists}\label{ss:dnl}\index{search!names!\texttt{dnl}}

\hypertarget{il:dnl}{\texttt{dnl}} searches\index{dictionary!search!\texttt{dnl}} and returns dictionary name lists. 
The entire path is searched for names and duplicates are removed. 
A negative option code requests a \emph{path order list.}\index{path order list} A path order list returns
 the objects in each directory in path order. 
 Raising, removing duplicates and sorting a path order list gives a 
 standard \texttt{dnl} list. \texttt{dnl} arguments follow the pattern: 

\begin{description}
\item $(n, \left\langle p, \left\langle d \right\rangle \right\rangle)$ \texttt{dnl 'str'}
%\item (n, $\left\langle$ p, $\left\langle$ d $\right\rangle$ $\right\rangle$) dnl 'str'
\item \emph{ where:}
\begin{description}
\item $n$ is one of \verb|0 1 2 3 4 |
\item optional $p$ is one of \verb|1 2 3 _1 _2 _3 |
\item optional $d$ is word name class or macro type
\end{description}
\end{description}

\begin{wordhead}
\monad & dnl zl \argsep clPstr \\
\end{wordhead}
\begin{lstlisting}[frame=single,framerule=0pt]
  NB. list all words on current dictionary path
  dnl '' 
  
  NB. list all words that begin with prefix 
  dnl 'prefix'
\end{lstlisting}

\begin{wordhead}
\dyad & iaObject dnl zl \argsep clPstr \\
      & (iaObject,iaOption) dnl zl \argsep clPstr \\
      & (iaObject,iaOption,iaQualifier) dnl zl  \\
      & (iaObject,iaOption,iaQualifier) dnl clPstr \\
\end{wordhead}
\begin{lstlisting}[frame=single,framerule=0pt]
  0 dnl ''  NB. all words (monad)
  1 dnl ''  NB. list all tests
  2 dnl ''  NB. list all groups
  3 dnl ''  NB. list all suites
  4 dnl ''  NB. list all macros
\end{lstlisting}

A word can appear in two dictionaries. 
When getting such a word the first 
path occurrence is the value returned. 
The second value is \emph{shadowed} by the first
as only one value can be retrieved.

\begin{lstlisting}[frame=single,framerule=0pt]  
  NB. match word names beginning with str
  0 1 dnl 'str'  
  
  NB. match word names containing the string str
  0 2 dnl 'str'  
  
  NB. match word names ending with string str 
  0 3 dnl 'str'  

  NB. words and macros have an optional third
  NB. item that denotes name class or type
  
  NB. adverb names beginning with str
  0 1 1 dnl 'str'  
  
  NB. verb names containing str
  0 2 3 dnl 'str'  
  
  NB. nouns ending with str
  0 3 0 dnl 'str'  

  NB. J macro names beginning with jscript
  4 1 21 dnl 'jscript' 

  NB. LaTeX macro names containing latex
  4 2 22 dnl 'latex'   
  
  NB. HTML macro names ending with html 
  4 3 23 dnl 'html' 
 \end{lstlisting}  

  A negative second item option code returns a path order list.\index{path order list}

\begin{lstlisting}[frame=single,framerule=0pt] 
  NB. words containing str (result is a list of lists)
  0 _2 3 dnl 'str' 
  
  NB. group names beginning with so
  2 _1 dnl 'so'   
  
  NB. suite names ending with str 
  3 _3 dnl 'str'   
\end{lstlisting}

\subsection{\texttt{doc} --- format comments}\label{ss:doc}

\hypertarget{il:doc}{\texttt{doc}} formats the leading comment block 
of explicit words, tests, group/suite headers and macros.  The comment block must follow J
 \texttt{scriptdoc}\index{scriptdoc} compatible conventions.  The comment style processed by \texttt{doc} is 
illustrated in the following example. More examples of \texttt{doc} 
formatting can be examined by displaying 
words in the distributed JOD dictionaries.\index{documentation!\texttt{doc}} 
Install the \href{http://www.jsoftware.com/jwiki/Addons/general/jodsource}{\texttt{jodsource}} 
addon to get the distributed JOD dictionaries.

\begin{lstlisting}[frame=single,framerule=0pt]
docexample0=: 3 : 0
NB.*docexample0 v-- the leading block of comments
NB. can be a scriptdoc compatible mess as far
NB. as formatting goes.
NB.
NB. However, if you run doc over
NB. a word in a JOD dictionary your 
NB.     mess is cleaned up. See below.
NB. monad: docexample uuHungarian
NB.
NB.   text below monad: and dyad: is left unformatted
NB.   this region is used to display example calls
J code from now on
) 
\end{lstlisting}

\begin{lstlisting}[frame=single,framerule=0pt]
docexample0=:3 : 0
NB.*docexample0 v-- the leading block of comments can be a
NB. scriptdoc compatible mess as far as formatting goes.
NB.
NB. However, if you run doc over a word in a JOD dictionary
NB. your mess is cleaned up. See below.
NB.
NB. monad: docexample uuHungarian
NB.
NB.   text below monad: and dyad: is left unformatted
NB.   this region is used to display example calls
j code from now on
) 
\end{lstlisting}

\begin{wordhead}
\monad & doc clName \\
\end{wordhead}
\begin{lstlisting}[frame=single,framerule=0pt]
  NB. format leading comment block 
  doc 'formatme' 
\end{lstlisting}

\begin{wordhead}
\dyad & iaObject doc clName  \\
      & (iaObject,iaOption) doc clName \\
\end{wordhead}
\begin{lstlisting}[frame=single,framerule=0pt]
  NB. format leading test script comments
  1 doc 'tidytest'
  
  NB. format J script macros - only J scripts formatted
  4 doc 'macroname'
  
  NB. group and suite headers
  2 1 doc 'thisgroup'
  3 1 doc 'thissuite'
\end{lstlisting} 
  
\texttt{doc} also formats the long document text of all dictionary objects.
Long document text differs from J script
text in that it is not prefixed with leading \verb|NB.|'s.
  
\begin{lstlisting}[frame=single,framerule=0pt]
  NB. format long documents
  0 9 doc 'word'
  1 9 doc 'test'
  2 9 doc 'group'
  3 9 doc 'suite'
  4 9 doc 'macro'
\end{lstlisting}  


\subsection{\texttt{dpset} --- set and change parameters}\label{ss:dpset}\index{parameters!\texttt{dpset}}

\hypertarget{il:dpset}{\texttt{dpset}} modifies dictionary parameters. JOD uses a 
variety of values that control putting, getting and generating objects. 
Parameters\index{dictionary!parameters} are stored in individual dictionaries and 
the master file. Master file\index{master file} parameters\index{parameters|see{dictionary}} are initially 
set from the \verb|jodparms.ijs| file, see appendix~\ref{ap:jodparms}, on
page~\pageref{ap:jodparms}, and cannot be reset without 
editing \verb|jodparms.ijs| and recreating the master file. Individual 
dictionary parameters can be changed at any time with \texttt{dpset}. \texttt{dpset} is permissive; 
it will allow parameters to be set to any value. \emph{Invalid values will 
crash JOD!} Before setting any values examine the \verb|jodparms.ijs| file. 
This file is used to set the default values of dictionary parameters. 

\emph{Note:} If you set an invalid parameter
 value you can recover using \texttt{dpset}'s \texttt{DEFAULTS} option. 

Not all dictionary parameters can be set by \texttt{dpset}. 
The parameters \texttt{dpset} can change are dictionary specific 
user parameters. There are a number of system wide parameters 
that are set in code and require script edits to change. 

If JOD, or the host OS crashes, the master file could be
 left in a state that makes it impossible to reopen dictionaries.
  \verb|dpset 'RESETME'| and \verb|dpset 'RESETALL'| clear read status codes in the master file.\index{master file}
   \texttt{RESETME} resets all dictionaries recently opened from the current machine. 
   \texttt{RESETALL} resets all dictionaries in the master file.  
   In the worst case you can rebuild the master file by:
\begin{quotation}
	\noindent\begin{enumerate}
	  \item Exiting J.
	  \item Deleting the files:
	\begin{description}
		\item \verb|~addons/general/jod/jmaster.ijf| 
		\item \verb|~addons/general/jod/jod.ijn| 
	\end{description}	
	  \item Restarting J.
	  \item Reloading JOD with: \verb|load 'general/jod'|
	\end{enumerate}  
\end{quotation}

\begin{wordhead}
\monad & dpset zl \argsep clName \argsep (clName;uuParm) \\
\end{wordhead}
\begin{lstlisting}[frame=single,framerule=0pt]
  NB. list all parameters and current values
  dpset ''  
  
  NB. restore default settings in put dictionary          
  dpset 'DEFAULTS'    

  NB. option names are case sensitive 
  NB. resets current machine dictionaries
  dpset 'RESETME'  
  
  NB. resets all dictionaries    
  dpset 'RESETALL'
\end{lstlisting}  

Note: if a JOD dictionary is being used 
by more than one task never use \texttt{RESETALL} unless 
you are absolutely sure you will not reset other tasks! 

\begin{lstlisting}[frame=single,framerule=0pt]
  NB. clears the put dictionary reference path
  dpset 'CLEARPATH'  
  
  NB. makes the current put dictionary read-only 
  dpset 'READONLY'    
  
  NB. makes the current put dictionary read-write
  dpset 'READWRITE'   
  
  NB. use ascii85 in dump scripts 
  dpset 'ASCII85';1   
  
  NB. retain age of objects in dump scripts
  dpset 'RETAINAGE';1
  
  NB. set J configured make script directory
  dpset 'ROOTFOLDER';'~USERROOT/mlsscripts/' 
\end{lstlisting}


\subsection{\texttt{ed} --- edit dictionary objects}\label{ss:ed}

\hypertarget{il:ed}{\texttt{ed}} fetches or generates dictionary objects and puts 
them in an edit\index{edit!\texttt{ed}} window for editing.\index{test suite}

\begin{wordhead}
\monad & ed clName \argsep blclNames \argsep btNameClassValue \\
\end{wordhead}
\begin{lstlisting}[frame=single,framerule=0pt]
  NB. retrieve word and place in edit window
  ed 'word'
  
  NB. put many words in edit window 
  ed ;:'many words edited' 
  
  NB. edit objects in (name,[class],value) tables
  'rc ncv'=. 4 get }. 4 1 21 dnl 'build'
  ed ncv
  
  NB. edit backup versions of a word in a single edit window
  'rc ncv'=. bget <;._1 ' me.12 me.09 me.08 me.02'
  ed ncv
  
  NB. edit all words in last backup in a single edit window
  'rc ncv'=. bget }. bnl ''
  ed ncv
\end{lstlisting} 

\begin{wordhead}
\dyad & iaObject ed clName \argsep blclNames \argsep btNameClassValue \\
      & (iaObject,iaOption) ed clPstr \\
\end{wordhead}
\begin{lstlisting}[frame=single,framerule=0pt]      
  NB. edit test
  1 ed 'test'
  
  NB. generate group and place in edit window    
  2 ed 'group'  
   
  NB. generate test suite and place in edit window
  3 ed 'suite'  
  
  NB. edit macro text 
  4 ed 'macro'  
   
  NB. edit group header text
  2 1 ed 'group' 
  
  NB. edit suite header text
  3 1 ed 'suite'  
\end{lstlisting}

\subsection{\texttt{et} --- put text into edit window}\label{ss:et} 

\hypertarget{il:et}{\texttt{et}} load character lists into edit\index{edit!\texttt{et}} windows.

\begin{wordhead}
\monad & et clText \\
\end{wordhead}
\begin{lstlisting}[frame=single,framerule=0pt]
  NB. put character data into edit window
  et 'put text in edit window' 

  NB. read text and put in edit window
  et (1!:1) <'c:\temp\text.txt' 
\end{lstlisting}

\subsection{\texttt{gdeps} --- group dependents}\label{ss:gdeps}

\hypertarget{il:gdeps}{\texttt{gdeps}} returns lists of global names\index{dependent section} 
in the dependent section of group and suite headers: see page~\pageref{ss:depsec}.

\begin{wordhead}
\monad & gdeps clGroup \\
\end{wordhead}
\begin{lstlisting}[frame=single,framerule=0pt]
  NB. globals in the dependent section of group jod
  gdeps 'jod'  
  
  NB. all dependent section globals in all groups
  gdeps&> }. grp ''
\end{lstlisting}

\begin{wordhead}
\dyad & iaOption gdeps clName \\
\end{wordhead}
\begin{lstlisting}[frame=single,framerule=0pt]
  NB. globals in the dependent section of suite testenv
  3 gdeps 'testenv'  
\end{lstlisting}

\subsection{\texttt{get} --- get objects}\label{ss:get}

\hypertarget{il:get}{\texttt{get}} retrieves dictionary objects and information 
about dictionary objects. 
There is a close correspondence between the arguments of \texttt{get} 
and \hyperlink{il:put}{\texttt{put}}, see page~\pageref{ss:put}. 
A basic JOD rule is that if you can \texttt{put} it you can \texttt{get} it.

\begin{wordhead}
\monad & get clName \argsep blclNames \\
\end{wordhead}
\begin{lstlisting}[frame=single,framerule=0pt]
  NB. get word and define in current locale
  get 'word'  
  
  NB. get a group  
  get }. grp '' 
\end{lstlisting}

\begin{wordhead}
\dyad & ilOptions get clName \argsep blclNames \argsep uuIgnore \\
      & clLocale get clName \argsep blclNames \\
\end{wordhead}
\begin{lstlisting}[frame=single,framerule=0pt]
  NB. get word (monad)
  0 get 'word'  
  
  NB. get words (monad) 
  0 7 get ;:'words are us' 
\end{lstlisting} 

  For words a character left argument is a target locale.
  
\begin{lstlisting}[frame=single,framerule=0pt]  
  NB. get into locale
  'locale' get ;:'hi ho into locale we go'  
  
  NB. allow numbered locales
  '666' get ;:'beast code'  
    
  NB. explain words
  0 8 get ;:'explain us ehh'  
  
  NB. word documentation
  0 9 get ;:'document or die' 
  
  NB. get word scripts without defining 
  0 10 get 'define';'not' 
\end{lstlisting} 
    
  Information about stored words can be retrieved with \texttt{get}.
  
\begin{lstlisting}[frame=single,framerule=0pt]
  NB. J name class of words
  0 12 get ;:'our name class' 
  
  NB. word creation dates
  0 13 get ;:'our creation'   
  
  NB. last word put dates
  0 14 get ;:'last change'    
  
  NB. word size in bytes
  0 15 get ;:'how big are we' 
  
  NB. get test scripts
  1 7 get 'i';'test';'it'  
     
  NB. test explanations
  1 8 get ;:'explain tests'   
  
  NB. test case documentation
  1 9 get 'radical'
  
  NB. get word names and timestamps 
  0 _14 get ;:'our time stamps'
  
  NB. get macro names and timestamps
  4 _14 get ;:'macro creation lastput'        
\end{lstlisting}         

  \texttt{get} fetches information about stored\index{testing!\texttt{get}} tests.
  
\begin{lstlisting}[frame=single,framerule=0pt]  
  NB. test creation dates
  1 13 get ;:'our creation'  
  
  NB. last test put dates 
  1 14 get ;:'last change' 
  
  NB. test size in bytes   
  1 15 get ;:'how big are we' 
  
  NB. get group scripts
  2 7 get ;:'groupies cool'  
  
  NB. get group explanation text
  2 8 get 'group';'explain'  
  
  NB. get group document text 
  2 9 get 'document'      
  
  NB. suite text    
  3 7 get ;:'this suites me'    
  
  NB. explain suites    
  3 8 get ;:'suites need comments'  
  
  NB. document suites
  3 9 get ;:'document your suites'  
  
  NB. get various macros
  4 get 'jmacro';'html';'latex'  
  
  NB. explain macros
  4 8 get ;:'macros need explaining'  
  4 9 get ;:'and documents too'
  
  NB. get dictionary documentation - ignores (y) argument
  5 get ''     
\end{lstlisting}


\subsection{\texttt{getrx} ---  get required to execute}\label{ss:getrx}

\hypertarget{il:getrx}{\texttt{getrx}} gets all the words required to execute words on \texttt{(y)}.
 
\emph{Warning:}  if  the words listed on  \texttt{(y)} refer to  object  or 
locale references\index{cross references!\texttt{getrx}}  this verb  returns  
an error because such words 
generally cannot be run out of context.

\begin{wordhead}
\monad & getrx clName \argsep blclNames \\
\end{wordhead}
\begin{lstlisting}[frame=single,framerule=0pt] 
  NB. load required words into base locale
  getrx 'stuffineed'       

  NB. get all words required to run many words 
  getrx ;:'stuff we need to run' 
\end{lstlisting}

\begin{wordhead}
\dyad & clLocale getrx clName \argsep blclNames \\
\end{wordhead}
\begin{lstlisting}[frame=single,framerule=0pt] 
  NB. load all required words into locale
  'locale' getrx ;:'load the stuff we need into locale' 
\end{lstlisting}


\subsection{\texttt{globs} --- global references}\label{ss:globs}

\hypertarget{il:globs}{\texttt{globs}} analyzes 
global\index{cross references!\texttt{globs}} references 
in words and tests. A global reference is a 
nonlocal J name where nonlocality is with respect to the current word's scope. 
Names with locale references, for example: 

\begin{lstlisting}[frame=single,framerule=0pt] 
   jread_jfiles_     NB. direct locale reference
   did__jd2          NB. indirect locale (object) reference 
   did_3_            NB. direct numbered local reference
   boo__hoo__too     NB. two levels of indirection 
   
   NB.  suffix nouns: jd2 hoo too
\end{lstlisting}
  
%\begin{enumerate}
%	\item \verb|jread_jfiles_|    \textcolor{CodeComment}{\texttt{\textsl{NB. direct locale reference}}} 
%	\item \verb|did__jd2|         \textcolor{CodeComment}{\texttt{\textsl{NB. indirect locale (object) reference}}}
%	\item \verb|did_3_|           \textcolor{CodeComment}{\texttt{\textsl{NB. direct numbered local reference}}}
%	\item \verb|boo__hoo__too|    \textcolor{CodeComment}{\texttt{\textsl{NB. two levels of indirection}}}
%\end{enumerate}
 
\noindent are treated like primitives. This makes it possible to define clean locale/object 
interfaces. In the case of indirect locale references the suffix noun(s) must exist 
to determine the name class of the word. This makes static name analysis difficult. 
By treating such references as ``primitives'' this problem is swept 
under the proverbial rug. 

For example the \texttt{jfiles} utility is often 
accessed with \textbf{\texttt{z}} locale definitions like: 

\begin{lstlisting}[frame=single,framerule=0pt] 
   jread_z_ =: jread_jfiles_   NB. z interface for jread
\end{lstlisting}

Words that use \texttt{jread} can call it without locale suffixes. 
For this case \texttt{globs} will detect the use of \texttt{jread} but will 
cease searching the call tree when it encounters \verb|jread_jfiles_|.

Globals referenced by test scripts are not stored because tests 
often manipulate their working environments in ways that make 
static name analysis unfeasible. \texttt{globs} is one of two verbs, 
(\texttt{globs}, \hyperlink{il:grp}{\texttt{grp}}), that create references. 
For \texttt{globs} to store 
references the word must be in the put dictionary,\index{put dictionary} all 
word references must exist on the path and the current 
path must match the put dictionary reference path. 

Finally, \texttt{globs} expects all assigned local and global word names to differ. If \texttt{globs}
finds names that are both globally and locally assigned it returns a mixed assignments error.

\begin{lstlisting}[frame=single,framerule=0pt]
mixedassignment=: 3 : 0

NB. I have a mixed global and local assignment. Usually, this is 
NB. poor name choice but there are good reasons for such assignments.

names=. ;:'we are going to be globals'
(names)=: <'clandestine global'
)
\end{lstlisting}

\noindent To override \texttt{globs} mixed assignment handling use the comment tag
 \textcolor{CodeComment}{\texttt{\textsl{(<:)=:}}}\index{comment tag!\texttt{(<:)=:}}.
If this tag is used anywhere in a word definition \texttt{globs} will ignore any 
mixed assignments in the tagged word.

\begin{lstlisting}[frame=single,framerule=0pt]
confusing=: 3 : 0

NB. allow mixed assignments (<:)=:
scopeless=. <'cis_scoping_is_so_un_woke'
(scopeless)=: scopeless 
)
\end{lstlisting}



\begin{wordhead}
\monad & globs clName \\
\end{wordhead}
\begin{lstlisting}[frame=single,framerule=0pt] 
  NB. list globals in locale word 
  globs 'word' 
\end{lstlisting}

\begin{wordhead}
\dyad & iaObject globs clName \\
\end{wordhead}
\begin{lstlisting}[frame=single,framerule=0pt] 
  NB. update referenced globals
  0 globs 'word' 

  NB. update all words in a group
  0 globs&> }. grp 'group' 

  NB. list global references in test text
  1 globs 'test'           

  NB. classify name references in locale word. 
  11 globs 'word'  
\end{lstlisting}


\subsection{\texttt{grp} --- create and modify groups}\label{ss:grp}

\hypertarget{il:grp}{\texttt{grp}} creates and modifies word groups\index{group!\texttt{grp}}
 and test\index{testing!\texttt{grp}} suites. A group is a list of objects. Operations on groups do not change 
 the objects that belong to groups. When a group is created the
 put dictionary's reference path\index{reference path} is compared to the current dictionary path. 
 If the paths do not match an error is returned and the group is not created. 

\begin{wordhead}
\monad & grp zl \argsep clName \argsep blclNames \\
\end{wordhead}
\begin{lstlisting}[frame=single,framerule=0pt] 
  NB. list all word groups (2 dnl '')
  grp ''    
  
  NB. list words in group    
  grp 'group'  
  
  NB. create empty group 
  grp 'soempty';''

  NB. create/reset group first name is the group name
  grp 'group';'list';'of';'group';'names'

  NB. has effect of emptying but not deleting group 
  grp <'group' 
\end{lstlisting}

\begin{wordhead}
\dyad & iaObject grp zl \argsep clName \argsep blclNames \\
\end{wordhead}
\begin{lstlisting}[frame=single,framerule=0pt] 
  NB. list all test suites (3 dnl '')
  3 grp ''  
  
  NB. list tests in suite   
  3 grp 'suite' 
  
  NB. (monad)
  2 grp 'group';'list';'of';'group';'names'  
  
  NB. create/reset suite
  3 grp 'suite';'list';'of';'test';'names'  
  
  NB. empty suite  
  3 grp <'suite' 
\end{lstlisting}


\subsection{\texttt{gt} --- get edit window text} 

Fetch text from edit\index{edit!\texttt{gt}} window.

\begin{wordhead}
\monad & gt zl \argsep clName \\
\end{wordhead}
\begin{lstlisting}[frame=single,framerule=0pt] 
  NB. returns text from the word.ijs edit window
  gt 'word'  

  NB. using gt to update a test and macro.
  1 put 'test';gt 'test' 

  4 put 'macro';21;gt 'macro' 
\end{lstlisting}


\subsection{\texttt{hlpnl} --- display short object descriptions}

\texttt{hlpnl} displays short object descriptions.\index{documentation!\texttt{hlpnl}}  

Short object descriptions\index{help!\texttt{hlpnl}} are always a good idea.  
If you cannot \emph{tersely} describe an object you probably don't 
understand it.  Short descriptions\index{test suite} 
are stored with \hypertarget{il:put}{\texttt{put}}.

\begin{wordhead}
\monad & hlpnl clName \argsep blclNames \\
\end{wordhead}
\begin{lstlisting}[frame=single,framerule=0pt] 
  NB. put short word description 
  0 8 put 'describeme';'briefly describe me' 

  NB. display short word description
  hlpnl 'describeme' 

  NB. display many descriptions
  hlpnl ;:'show our short word descriptions'

  NB. describe all the words in a group
  hlpnl }. grp 'groupname'

  NB. describe all the words called by a word
  hlpnl allrefs <'wordname'

  NB. describe all dictionary words
  hlpnl }. dnl ''
\end{lstlisting}

\begin{wordhead}
\dyad & iaObject hlpnl clName \argsep blclNames \\
\end{wordhead}
\begin{lstlisting}[frame=single,framerule=0pt] 
  NB. display short word description (monad)
  0 hlpnl 'word' 

  NB. display test, group, suite, macro descriptions
  1 hlpnl 'testname'
  2 hlpnl 'groupname'
  3 hlpnl 'suitename'
  4 hlpnl 'macroname'

  NB. describe a test suite
  3 hlpnl }. 3 dnl 'testsuite'

  NB. describe a group
  2 hlpnl }. 2 dnl 'groupname'

  NB. describe macro scripts with prefix prj
  4 hlpnl }. 4 dnl 'prj'
\end{lstlisting}


\subsection{\texttt{jodage} --- age of JOD objects}  

\hypertarget{il:jodage}{\texttt{jodage}} returns the age of JOD objects.  When an object is 
put into a dictionary the date\index{dates} is recorded. 

The monad returns the age of words and the dyad returns the age of other objects.  
JOD dates are stored in a fractional day \verb|yyyymmdd.fd| floating point format.\footnote{JOD times
are derived from \emph{local} computer clock times. \texttt{UTC} is not used.
}    

\begin{wordhead}
\monad & jodage clWord \argsep blclWords \\
\end{wordhead}
\begin{lstlisting}[frame=single,framerule=0pt] 
  NB. show age of (jodage)  
  jodage 'jodage' 

  NB. age of all group words 
  jodage }. grp 'bstats' 
\end{lstlisting}

\begin{wordhead}
\dyad & ia jodage clWord \argsep blclNames \\
\end{wordhead}
\begin{lstlisting}[frame=single,framerule=0pt] 
  NB. age of all test scripts  
  1 jodage }. 1 dnl ''   

  NB. age of group script  
  2 jodage 'mygroup'     

  NB. age of all macro scripts  
  4 jodage }. 4 dnl ''   
\end{lstlisting}


\subsection{\texttt{jodhelp} --- display help}\label{ss:jodhelp}  

\hypertarget{il:jodhelp}{\texttt{jodhelp}} displays JOD documentation\index{help!\texttt{jodhelp}}.  
The monad starts the configured J PDF reader and displays \verb|jod.pdf|. 

\verb|jod.pdf| is omitted from the core JOD addon \texttt{JAL} package to
reduce download size. Installing the addon \texttt{general/joddocument} 
saves \verb|jod.pdf| in the directory searched by \texttt{jodhelp}.
\begin{lstlisting}[frame=single,framerule=0pt]
  NB. JOD addon PDF document directory
  ~addons/general/joddocument/pdfdoc
\end{lstlisting}

\begin{wordhead}
\monad & jodhelp uuIgnore \\
\end{wordhead}
\begin{lstlisting}[frame=single,framerule=0pt] 
  NB. display help - requires general/joddocument addon
  jodhelp 0   

  NB. argument is ignored
  jodhelp 'I don''t hear you!' 
\end{lstlisting}  


\subsection{\texttt{lg} ---  make and load group}\label{ss:lg}

\hypertarget{il:lg}{\texttt{lg}} assembles\index{scripts!generation!\texttt{lg}} 
and loads JOD group\index{group!\texttt{lg}} scripts.  The monad loads without the postprocessor 
script and the dyad loads with the postprocessor.

The postprocessor\index{postprocessor} is a JOD macro script that is associated with a group.
If a group\index{group!\texttt{lg}} is named \texttt{numtutils} the associated postprocessor 
is named \verb|POST_numutils|. The prefix \verb|POST_| labels J macro scripts as postprocessors.\footnote{A macro
must be coded as a J script, (code 21), to be used as a postprocessor.
} The postprocessor is appended to generated group scripts and is often used to start systems.

\begin{wordhead}
\monad & lg clGroup \\
\end{wordhead}
\begin{lstlisting}[frame=single,framerule=0pt] 
  NB. make and load group without postprocessor 
  lg 'groupname'
\end{lstlisting}

\begin{wordhead}
\dyad & iaOption lg clGroup \\
\end{wordhead}
\begin{lstlisting}[frame=single,framerule=0pt] 
  NB. monad
  2 lg 'groupname' 

  NB. define a group postprocessor macro script
  NB. 21 identifies macro text as an arbitrary J script
  4 put 'POST_groupname';21;'smoutput ''hello world'''

  NB. make and load appending postprocessor
  lg~ 'groupname' 
\end{lstlisting}


\subsection{\texttt{locgrp} --- list groups/suites with word/test}\label{ss:locgrp}\index{search!groups!\texttt{locgrp}}

\texttt{locgrp} lists groups\index{group!\texttt{locgrp}} and suites with word or test \texttt{(y)}.   
A word or test can belong to many groups or suites. 

\begin{wordhead}
\monad & locgrp clName \\
\end{wordhead} 
\begin{lstlisting}[frame=single,framerule=0pt]  
  NB. list all groups that contain myword
  locgrp 'myword' 

  NB. list all suites that contain this test
  locgrp 'thistest' 
\end{lstlisting}


\subsection{\texttt{make} --- generates dictionary scripts}\label{ss:make}

\hypertarget{il:make}{\texttt{make}} generates J 
scripts\index{scripts!generation!\texttt{make}} from objects stored
in\index{dictionary!dump!\texttt{make}} dictionaries.
The generated scripts can be returned as results or written to file: 
see subsection~\ref{ss:genscripts} on page~\pageref{ss:genscripts}.

Generated scripts are stored in the standard dump,\index{scripts!dump!\texttt{make}} 
script and suite subdirectories.  Monadic \texttt{make} dumps all the objects 
on the current path to a J script file.  The dump file is a single J script 
that can be used to rebuild dictionaries. 

\texttt{make} uses the reference path\index{reference path} to generate words, 
tests, groups and suites.
When generating groups and suites \texttt{make} returns an error if the current 
path does not match the reference path. By default dyadic \texttt{make} generates 
objects that exist in the current put dictionary. 
This can be overridden with a negative option code. 

\begin{wordhead}
\monad & make zl \argsep clDumpfile \\
\end{wordhead}
\begin{lstlisting}[frame=single,framerule=0pt] 
  NB. Dump objects on current path 
  NB. to put dictionary dump directory.
  NB. The name of the put dictionary is 
  NB. used as the dump file name.
  make '' 

  NB. dump to specified Windows file
  make 'c:/dump/on/me.ijs'
  
  NB. linux and mac paths must begin with /
  make '/home/john/temp/joddumps/metoo.ijs'
\end{lstlisting}

\begin{wordhead}
\dyad & iaObject make zl \argsep clName \argsep blclNames \\
      & (iaObject,iaOption) make clName \\
\end{wordhead}
\begin{lstlisting}[frame=single,framerule=0pt] 
  0 make ;:'an arbitrary list of words into a script'
  0 2 make ;:'generate a character list script result' 

  NB. make J script that defines a group
  2 make 'group' 
  
  NB. make J script that defines a suite 
  3 make 'suite' 
\end{lstlisting}

An option code controls whether 
results are written to file, (1 default),
or returned, (2 return), for word lists, groups and suites.
Default dictionary file locations are the 
subdirectories created by \hyperlink{il:newd}{\texttt{newd}}: see page~\pageref{ss:newd}.

\begin{lstlisting}[frame=single,framerule=0pt] 
  NB. make and return group script
  2 2 make 'group'  
  
  NB. make put dictionary suite script and write to file 
  3 1 make 'suite'  

  NB. make and file group script. The group does not
  NB. have to exist in the put dictionary but can
  NB. occur anywhere on the path. 
  2 _1 make 'group'

  NB. make suite script and write to file 
  3 _1 make 'suite' 
\end{lstlisting}


\subsection{\texttt{mls} --- make load script}\label{ss:mls} 

\hypertarget{il:mls}{\texttt{mls}} generates J 
load\index{load scripts} scripts.\index{scripts!generation!\texttt{mls}}  
The generated script is added 
to the current user's start up script
\begin{verbatim}
  ~config/startup.ijs
\end{verbatim}
and inserted in the session's \verb|Public_j_| table.\footnote{On J 
\texttt{6.0x} systems the public script noun is named \texttt{PUBLIC\_j\_}.}

An \verb|mls| load script\index{load scripts} is independent of JOD and can be used 
like any other J load script, for example:

\begin{lstlisting}[frame=single,framerule=0pt]
  load 'mlsmademe'
\end{lstlisting}

The generated script can be written to file or returned. 
Generated scripts are stored in the put dictionary script subdirectory.  
\texttt{mls} appends any postprocessor\index{postprocessor} to the generated script: 
see subsection~\ref{ss:genscripts} Generated Script Structure, on page~\pageref{ss:genscripts}.

\begin{wordhead}
\monad & mls clGroupname \\
\end{wordhead}
\begin{lstlisting}[frame=single,framerule=0pt] 
  NB. add a postprocessor script for (addgroup) 
  postproc=. 'smoutput ''''this is a post processor'''''
  4 put 'POST_appgroup';JSCRIPT_ajod_;postproc  

  NB. generate group script with 
  NB. postprocessor and add to startup.ijs
  mls 'appgroup' 
  
  NB. load group - postprocessor runs
  load 'appgroup'
  
  NB. redirect generated script by setting 
  NB. ROOTFOLDER to a configured J folder 
  jpath '~user/jodroot' 
  dpset 'ROOTFOLDER';'~user/jodroot'
  mls 'appgroup'
\end{lstlisting}

\begin{wordhead}
\dyad & iaOption mls clGroupname \\
\end{wordhead}
\begin{lstlisting}[frame=single,framerule=0pt] 
  NB. make J script file but do 
  NB. not add to startup.ijs
  0 mls 'bstats' 

  NB. monad 
  1 mls 'bstats' 

  NB. return generated script as result 
  NB. does not add to startup.ijs  
  2 mls 'bstats' 
\end{lstlisting}

\subsection{\texttt{mnl} --- master name lists}\label{ss:mnl}\index{search!names!\texttt{mnl}}

\hypertarget{il:mnl}{\texttt{mnl}} searches\index{dictionary!search!\texttt{mnl}} 
the name lists of all registered dictionaries, see \hyperlink{il:regd}{\texttt{regd}} on page~\pageref{ss:regd}.
 \texttt{mnl} does not require open dictionaries. 
\texttt{mnl}'s arguments mirror \hyperlink{il:dnl}{\texttt{dnl}} with the 
exception of negative option codes, see page~\pageref{ss:dnl}.  
The primary use of \texttt{mnl} is looking for duplicate names. Duplicate names limit the utility of words
and make it difficult to use dictionaries on arbitrary paths.
If the dictionaries \texttt{d0}, \texttt{d1} and \texttt{d2}
contain common names the paths resulting from the following commands 
are not necessarily the same.

\begin{lstlisting}[frame=single,framerule=0pt] 
  od ;:'d0 d1 d2' 
  od ;:'d2 d0 d1'   
\end{lstlisting}

\begin{wordhead}
\monad & mnl zl \argsep clPstr \\
\end{wordhead}
\begin{lstlisting}[frame=single,framerule=0pt]
  NB. list all words in all registered dictionaries
  mnl '' 

  NB. list all words in all dictionaries that begin with prefix
  mnl 'prefix' 
\end{lstlisting}

\begin{wordhead}
\dyad & iaObject mnl zl \argsep clPstr \\
      & (iaObject,iaOption) mnl zl \argsep clPstr \\
      & (iaObject,iaOption,iaQualifier) mnl zl  \\
      & (iaObject,iaOption,iaQualifier) mnl clPstr \\
\end{wordhead}
\begin{lstlisting}[frame=single,framerule=0pt]
  0 mnl ''  NB. all words in all dictionaries (monad)
  1 mnl ''  NB. all tests in all dictionaries
  2 mnl ''  NB. all groups in all dictionaries
  3 mnl ''  NB. all suites in all dictionaries
  4 mnl ''  NB. all macros in all dictionaries
  
  NB. match word names beginning with str
  0 1 mnl 'str'  
  
  NB. match word names containing the string str
  0 2 mnl 'str'  
  
  NB. match word names ending with string str 
  0 3 mnl 'str'  

  NB. words and macros have an optional third
  NB. item that denotes name class or type
  
  NB. adverb names beginning with str
  0 1 1 mnl 'str'  
  
  NB. verb names containing str
  0 2 3 mnl 'str'  
  
  NB. nouns ending with str
  0 3 0 mnl 'str'  

  NB. J macro names beginning with jscript
  4 1 21 mnl 'jscript' 

  NB. LaTeX macro names containing latex
  4 2 22 mnl 'latex'   
  
  NB. HTML macro names ending with html 
  4 3 23 mnl 'html' 
\end{lstlisting}

A negative second item option code returns duplicate names.

\begin{lstlisting}[frame=single,framerule=0pt] 
  NB. duplicate words containing str
  0 _2 3 mnl 'str' 
  
  NB. duplicate group names beginning with so
  2 _1 mnl 'so'   
  
  NB. duplicate suite names ending with str
  3 _3 mnl 'str'   
\end{lstlisting}

\subsection{\texttt{newd} --- create a new dictionary}\label{ss:newd} 

 \hypertarget{il:newd}{\texttt{newd}} creates\index{dictionary!create!\texttt{newd}} 
 a new dictionary. 
 Dictionary creation generates
 a set of files in a standard dictionary directory structure:
 see Figure~\ref{eps:joddirs} on page~\pageref{eps:joddirs}.
  The root directory, dictionary name, and optional dictionary 
 documentation can be specified. All other dictionary 
 creation parameters are taken from the master file.
 
\begin{wordhead}
\monad & newd clDictionary \\
       & newd (clDictionary;clPath) \\
       & newd (clDictionary;clPath;clDocumentation) \\
\end{wordhead}
\begin{lstlisting}[frame=single,framerule=0pt] 
  NB. if no location is specified the dictionary
  NB. is created in the default directory
  newd 'makemydictionary'
  
  NB. create with name in location 
  newd 'new';'c:/location/' 
  newd 'deep';'d:/we/can/root/dictionaries/down/deep'
  
  NB. on linux create dictionaries in your $HOME directory
  newd 'homeboy';'/home/john/where/the/dictionaries/are'

  NB. optional third item is dictionary documentation
  newd 'new';'c:/location/';'Dictionary documentation ...' 
\end{lstlisting}

\subsection{\texttt{notgrp} --- not grouped}

\texttt{notgrp} list words or tests from \texttt{(y)} that are not in groups or
suites. Useful for finding loose ends and dead code.

\begin{wordhead}
\monad & notgrp clName \argsep blclNames \\
\end{wordhead}
\begin{lstlisting}[frame=single,framerule=0pt] 
  NB. recent ungrouped words
  notgrp }. revo ''  
  
  NB. all ungrouped words
  notgrp }. dnl ''  
\end{lstlisting}

\begin{wordhead}
\dyad & iaOption notgrp clName \argsep blclNames \\
\end{wordhead}
\begin{lstlisting}[frame=single,framerule=0pt]
  NB. ungrouped words (monad)
  2 notgrp }. dnl ''    

  NB. tests that are not in suites
  3 notgrp }. 1 dnl ''
\end{lstlisting}

\subsection{\texttt{nw} --- edit a new word}\label{ss:nw} 

\hypertarget{il:nw}{\texttt{nw}} edits\index{edit!\texttt{nw}} a new  word in 
an edit window using JOD format conventions. \texttt{nw} assumes the
new word is  \emph{explicit} but you can edit the default text to define \emph{tacit} words. 
\texttt{nw} will append the 
character list \texttt{DOCUMENTCOMMAND} to the text placed in an edit window.  
This allows the user to define an arbitrary script that is run when a word is defined.\footnote{Key combinations
like \texttt{CTRL-W} that run editor scripts depend on the J platform.
}

\begin{wordhead}
\monad & nw clName \\
\end{wordhead}
\begin{lstlisting}[frame=single,framerule=0pt] 
  NB. open an edit window with an explicit newword
  nw 'newword'  

  NB. define a script that is run when the J editor
  NB. window is run. {~N~} placeholders are replaced
  NB. with the name of the new word

  DOCUMENTCOMMAND_ijod_=: 0 : 0
  smoutput pr '{~N~}'
  )

  NB. edit a word with DOCUMENTCOMMAND
  nw 'placeholdername'
\end{lstlisting}

\begin{wordhead}
\dyad & iaNameclass nw clName \\
\end{wordhead}
\begin{lstlisting}[frame=single,framerule=0pt]
  NB. edit an explict adverb
  1 nw 'adverb'   

  NB. edit an explict conjunction
  2 nw 'conjunction'  

  NB. edit an explicit text noun
  0 nw 'text'  

  NB. edit an explicit word (monad)
  3 nw 'word'  

  NB. edit dyadic word
  4 nw 'dyad'  
\end{lstlisting}

\subsection{\texttt{nt} --- edit a new test}\label{ss:nt}

\hypertarget{il:nt}{\texttt{nt}} edits\index{edit!\texttt{nt}} a new test script in an edit window. 
\texttt{nt} looks for the test script \texttt{teststub} on the path and inserts 
\texttt{teststub} text in the edit window. \texttt{teststub} allows users to define dictionary specific
custom script formats.

\texttt{nt} searches \texttt{teststub} text and replaces the markers
\verb|{~T~}|, \verb|{~A~}|, \verb|{~D~}| and \verb|{~SD~}| with optional
\emph{title}, \emph{author}, \emph{date}, (\texttt{yyyymondd} format), and 
\emph{short date}, (\texttt{yymondd} format). The dyad makes additional 
delimited string replacements after processing
markers.

\begin{wordhead}
\monad & nt clTestName \\
\end{wordhead}
\begin{lstlisting}[frame=single,framerule=0pt] 
  NB. open an edit window
  nt 'newtest'  

  testheader=: 0 : 0
  NB.*{~T~} t-- custom test script header
  NB. 
  NB. This is my custom test script header.
  NB. The {~T~} strings are replaced with test
  NB. name and creation time {~D~}
  NB.
  NB. author: {~A~}
  NB. created: {~D~}
  ) 

  NB. save custom header in put dictionary
  1 put 'teststub';testheader
  
  NB. CLASSAUTHOR replaces {~A~}
  CLASSAUTHOR_ijod_=: 'Mansong Hydrogen'

  NB. edit a new test using the custom header
  NB. where title {~T~} is set from the (y) argument
  nt 'customtest'
\end{lstlisting}

\begin{wordhead}
\dyad & clReplacements nt clTestName \\
\end{wordhead}
\begin{lstlisting}[frame=single,framerule=0pt]
  NB. apply string replacements, delimited by first
  NB. character, to teststub text then open edit window
  '#I#can#CHANGE#you' nt 'newtest' 
\end{lstlisting}


\subsection{\texttt{od} --- open dictionaries}\label{ss:od}

 \hypertarget{il:od}{\texttt{od}} opens\index{dictionary!open!\texttt{od}}
 dictionaries. Open dictionaries are appended to the
 path in the order they are opened. Dictionaries can be opened \texttt{READWRITE}
 (default) or \texttt{READONLY}. Only one J task can open 
 a dictionary \texttt{READWRITE}.  
 Any number of tasks can open a dictionary \texttt{READONLY}. 
 If any task has a dictionary open \texttt{READONLY} it can only 
 be opened \texttt{READONLY} by other tasks. If a 
 dictionary is opened \texttt{READWRITE} by a task it 
 cannot be opened by other dictionary tasks. This harsh protocol
  insures that only one task can update a dictionary. 

The first dictionary on the search path is special!  
It is the only dictionary that can be updated by JOD verbs. 
Because most updates are puts the first dictionary is called the \emph{put dictionary}.

\begin{wordhead}
\monad & od zl \argsep clDictionary \argsep blclDictionaries \\
\end{wordhead}
\begin{lstlisting}[frame=single,framerule=0pt] 
  NB. list registered dictionaries 
  od '' 

  NB. open read/write
  od 'dictionary'   
  
  NB. opens d(i) read/write 
  od 'd1';'d2';'d3'  
\end{lstlisting}

\begin{wordhead}
\dyad & iaOption od zl \argsep clDictionary \argsep blclDictionaries \\
\end{wordhead}
\begin{lstlisting}[frame=single,framerule=0pt] 
  NB. list registered dictionaries (monad)
  1 od '' 
  
  NB. close all open dictionaries (related to did 4) 
  3 od '' 

  NB. open read/write (monad)
  1 od 'dictionary' 
  
  NB. open read only and append to any path 
  2 od 'dictionary'   

  NB. open d(i) read only and append to any path
  2 od 'd1';'d2';'d3' 
  
  NB. close dictionaries and remove from path
  3 od ;:'d0 d1 d2'  
  
  NB. all dictionary root directories
  4 od ''  
              
  NB. list all dictionaries as regd script
  5 od ''             
\end{lstlisting}


\subsection{\texttt{packd} --- backup and pack dictionaries}\label{ss:packd}

\hypertarget{il:packd}{\texttt{packd}}\index{backup!\texttt{packd}} removes all unused space 
from dictionary files by copying active components to new files.
 After the \texttt{packd} operation is complete the new dictionary files 
 are renamed to match the original files. During the copy operation directories 
 are checked against the items in dictionary files. If 
 a directory data discrepancy is detected
 the pack operation ends with an error. Old files are renamed 
 with an increasing sequential backup number prefix, e.g.: \verb|13jwords.ijf| and 
 retained in the backup subdirectory. If a \texttt{packd} operation succeeds 
 the backup dictionary has no directory data inconsistencies. 

A \texttt{packd} operation can be reversed with \hyperlink{il:restd}{\texttt{restd}}.
Individual objects in backups can be retrieved with \hyperlink{il:bget}{\texttt{bget}}, see page~\pageref{ss:bget}.
There is no JOD facility for deleting  backup files. To erase backup files use OS facilities. 

The read/write status of a dictionary is recorded in 
the master file. JOD assumes all tasks point to the same master file.\index{master file}\footnote{There is one
important exception to the single \texttt{jmaster.ijf} rule. \texttt{READONLY}
dictionaries can be safely registered in different master files. This allows 
easy sharing of \texttt{READONLY} library dictionaries on network drives with
the standard JOD/JAL setup.
} 

\begin{wordhead}
\monad & packd clDictionary \\
\end{wordhead}
\begin{lstlisting}[frame=single,framerule=0pt] 
  NB. packd requires an open READWRITE dictionary
  od 'dictionary'   
   
  NB. reclaim unused file space in dictionary
  NB. and retain original files as a backup 
  packd 'dictionary' 
\end{lstlisting}                  


\subsection{\texttt{put} --- store objects in dictionary}\label{ss:put}

The \hypertarget{il:put}{\texttt{put}}
 verb stores objects in the put dictionary\index{put dictionary}.
 It can store words, tests\index{testing!\texttt{put}}, groups, suites and macros. 
 As a general rule: if something can be stored with \texttt{put} 
 it can be retrieved by \hyperlink{il:get}{\texttt{get}}: see page~\pageref{ss:get}.

\begin{wordhead}
\monad & put clName \argsep blclNames \\
\end{wordhead}
\begin{lstlisting}[frame=single,framerule=0pt] 
  NB. default is put words from base locale 
  put 'word' 
  
  NB. store all base locale verbs in dictionary
  put nl 3
\end{lstlisting} 
  
\begin{wordhead}
\dyad & iaObject put clName \argsep blclNames \\
      & iaObject put clDocument \argsep btNvalues \\
      & clLocale put clName \argsep blclNames \argsep btNvalues \\
      & (iaObject,iaQualifier) put clName \argsep blclNames  \\
      & (iaObject,iaQualifier) put clName \argsep btNvalues \\
\end{wordhead}
\begin{lstlisting}[frame=single,framerule=0pt] 
  NB. put words (monad)
  0 put ;:'w0 w1 w2 w3 w4'    
  
  NB. put words from specified locale
  'locale' put 'w0';'w2';'w3' 
  
  NB. numbered locales 
  '99' put 'word'             

  NB. put explain/document text
  NB. words must exist in dictionary
  0 8 put (;:'w0 w1'),.('text ...';'text ...')
  0 9 put (;:'w0 w1'),.('text ...';'text ...') 

  NB. put words from name class value table
  0 10 put ('w0'; 'w1'),.(3;3),.'code0 ...';'code1 ...' 

  NB. put tests from name value table
  1 put (;:'t0 t1'),.('text ...';'text ...') 

  NB. put test explain/document text
  1 8 put (;:'t0 t1'),.('text ...';'text ...')
  1 9 put (;:'t0 t1'),.('text ...';'text ...') 
\end{lstlisting}

A group or suite header\index{scripts!headers!\texttt{put}} script is an arbitrary J script 
that preceeds the code generated by \hyperlink{il:make}{\texttt{make}} on page~\pageref{ss:make}.

\begin{lstlisting}[frame=single,framerule=0pt] 
  NB. put group header scripts from name,value table
  2 put (;:'g0 g1'),.('text ...';'text ...') 
  
  NB. group header scripts can be put 
  NB. with 2 1 as well - maintains put/get symmetry 
  2 1 put (;:'g0 g1'),.('text ...';'text ...') 

  NB. put group explain/document text
  2 8 put (;:'g0 g1'),.('text ...';'text ...')
  2 9 put (;:'g0 g1'),.('text ...';'text ...') 

  NB. put suite header scripts from name value table
  3 put (;:'s0 s1'),.('text ...';'text ...') 
  3 1 put (;:'s0 s1'),.('text ...';'text ...') 

  NB. put suite explain/document text
  3 8 put (;:'s0 s1'),.('text ...';'text ...')
  3 9 put (;:'s0 s1'),.('text ...';'text ...') 

  NB. put macro scripts from name, type, value table
  
  NB. J scripts - can be run with (rm)
  4 put (;:'m0 m1'),.(21;21),.('text ...';'...') 
\end{lstlisting}
 
The text types \texttt{LaTeX, HTML, XML, ASCII, BYTE, MARKDOWN, UTF8}
are stored as J character lists. 

\begin{lstlisting}[frame=single,framerule=0pt]   
  NB. LaTeX 
  4 put (;:'m0 m1'),.(22;22),.('text ...';'...') 
  
  NB. HTML 
  4 put (;:'m0 m1'),.(23;23),.('text ...';'...') 

  NB. XML 
  4 put (;:'m0 m1'),.(24;24),.('text ...';'...') 
  
  NB. plain ASCII text 
  4 put (;:'m0 m1'),.(25;25),.('text ...';'...') 
  
  NB. list of arbitrary bytes
  4 put (;:'m0 m1'),.(26;26),.('bytes ...';'...') 
  
  NB. UTF-8 unicode text 
  4 put (;:'m0 m1'),.(28;28),.('utf8 text ...';'...') 

  NB. put macro explain/document text
  4 8 put (;:'m0 m1'),.('text ...';'text ...')
  4 9 put (;:'m0 m1'),.('text ...';'text ...') 
  
  NB. update dictionary documentation text
  5 put 'go ahead document this dictionary'
  
  NB. dictionary documentation is controlled 
  NB. by DOCUMENTDICT with default 1
  dpset 'DOCUMENTDICT';0
  5 put 'this will not be stored' 
\end{lstlisting}
  
Finally \texttt{put} can store timestamps in the form returned by \hyperlink{il:get}{\texttt{get}}.

\begin{lstlisting}[frame=single,framerule=0pt]  
  NB. push yyyymmdd.fd lastput dates 15 years forward
  'rc nts'=: 0 _14 get }. dnl 'new'
  0 _14 put (<0 150000 + "0 1 ;1{nts) 1} nts 
\end{lstlisting}                  


\subsection{\texttt{regd} --- register dictionaries}\label{ss:regd}

\hypertarget{il:regd}{\texttt{regd}} registers\index{registration!\texttt{regd}} and unregisters 
 dictionaries in the master file.
 
 A dictionary is a set of files in a standard directory structure: see page~\pageref{ss:joddirs}. 
 The \hyperlink{il:newd}{\texttt{newd}} verb creates JOD directories and files.
  There is no JOD verb that destroys dictionaries; actual 
  deletion of dictionary files and directories must be done using 
  other means. However, you can unregister a dictionary. 
  When a dictionary is unregistered it is removed from the main 
  dictionary directory in the master file. It will no longer appear 
  on \hyperlink{il:od}{\texttt{od}} lists and will no longer be accessible with JOD interface verbs. 
  Conversely, you can also register dictionaries with \texttt{regd}. 

\begin{wordhead}
\monad & regd (clDictionary;clPath;clDocumentation) \\
\end{wordhead}
\begin{lstlisting}[frame=single,framerule=0pt] 
  NB. register dictionary with name 
  NB. directory and dictionary must exist
  regd 'name';'c:/location/' 

  NB. register linux dictionary with optional documentation
  regd 'name';'/home/john/location/';'Documentation text' 
\end{lstlisting} 

\begin{wordhead}
\dyad & iaOption regd clDictionary \\
\end{wordhead}
\begin{lstlisting}[frame=single,framerule=0pt] 
  NB. unregistering a dictionary does not delete files 
  3 regd 'name' 

  NB. regd can be used to rename dictionaries 
  NB. and update dictionary documentation
  
  NB. unregister
  'name path' =. _2 {. 3 regd 'badname'  
  
  NB. re-register with new name and documentation 
  doc =. 'brand spanking new documenation' 
  regd 'goodname';path;doc 
\end{lstlisting}                  


\subsection{\texttt{restd} --- restore backup dictionaries}\label{ss:restd} 

\hypertarget{il:restd}{\texttt{restd}}  restores the last backup\index{backup!\texttt{restd}} created by
 \hyperlink{il:packd}{\texttt{packd}}. 

\begin{wordhead}
\monad & restd clDictionary \\
\end{wordhead}
\begin{lstlisting}[frame=single,framerule=0pt] 
  NB. open dictionary READWRITE 
  NB. must be first dictionary on the path
  od 'lastbackup' [ 3 od '' 

  NB. restore last dictionary backup
  restd 'lastbackup' 
\end{lstlisting}                   


\subsection{\texttt{revo} --- list recently revised objects} 

\hypertarget{il:revo}{\texttt{revo}} lists recently recently revised objects. 
Only put dictionary objects can be revised and only 
\hyperlink{il:put}{\texttt{put}} operations are considered revisions. 

\begin{wordhead}
\monad & revo zl \argsep clName \\
\end{wordhead}
\begin{lstlisting}[frame=single,framerule=0pt] 
  NB. all put dictionary words in last put order
  revo ''   
    
  NB. revised words with names beginning with boo  
  revo 'boo'  
\end{lstlisting}  

\begin{wordhead}
\dyad & iaObject revo zl \argsep clName \\
\end{wordhead}
\begin{lstlisting}[frame=single,framerule=0pt] 
  NB. list all revised tests
  1 revo ''  
  
  NB. revised suites with names prefixed by boo 
  3 revo 'boo'  
\end{lstlisting}                  


\subsection{\texttt{rm} --- run macros}\label{ss:rm}
 
A JOD macro\index{macros!\texttt{rm}} is an arbitrary J script.  \hypertarget{il:rm}{\texttt{rm}} 
fetches J macro scripts and runs them. \texttt{rm} will only run J, code 21, macros; other types return errors.  

\texttt{rm} sets the current locale\index{locale!base} to \texttt{base} and starts 
executing macro scripts in \texttt{base}. 

\begin{wordhead}
\monad & rm cl \argsep blclNames \\
\end{wordhead}
\begin{lstlisting}[frame=single,framerule=0pt] 
  NB. run J macro 
  rm 'macro' 

  NB. run macros with names starting with DoUs
  rm }. dnl 'DoUs' 
\end{lstlisting}   

\begin{wordhead}
\dyad & iaOption rm zl \argsep clName \argsep blclNames \\
\end{wordhead}
\begin{lstlisting}[frame=single,framerule=0pt] 
  NB. run J script and suppress output 
  1 rm 'quiet' 

  NB. note the repeat 
  1 rm ;:'run silent run deep' 
\end{lstlisting}                  


\subsection{\texttt{rtt} --- run tautology tests}\label{ss:rtt}

\hypertarget{il:rtt}{\texttt{rtt}} runs tautology\index{tautology} 
test\index{testing!\texttt{rtt}} scripts stored in JOD dictionaries. 

J has a built in test facility see: \verb|(0!:2)| and \verb|(0!:3)|.  
These foreigns run scripts and stop if the result deviates 
from arrays of \texttt{1}'s.  This facility is used by J's developers 
and \texttt{rtt} applies it to dictionary test scripts.\index{test suite}  
  
\texttt{rtt} starts scripts in the \texttt{base} locale.\index{locale!base} 

\begin{wordhead}
\monad & rtt clName \argsep blclNames \\
\end{wordhead}
\begin{lstlisting}[frame=single,framerule=0pt] 
  NB. run test script as a tautology
  rtt 'tautologytest'    
  
  NB. run all tautology tests in a suite 
  rtt }. 3 grp 'testsuite'  
\end{lstlisting}  

\begin{wordhead}
\dyad & iaOption rtt clName \argsep blclNames \\
\end{wordhead}
\begin{lstlisting}[frame=single,framerule=0pt] 
  NB. same as monad
  0 rtt 'tautologytest'    
  
  NB. run tautology test and suppress output 
  1 rtt 'silenttautology'  
  
  NB. run test as plain script
  2 rtt 'plaintest'        
  
  NB. generate test suite and run as tautology 
  3 rtt 'suitename'  
   
  NB. generate test suite and run as silent tautology      
  4 rtt 'silentsuite'      
\end{lstlisting}   

\subsection{\texttt{rxs} --- regular expression search}\label{ss:rxs}\index{search!text!\texttt{rxs}}

\hypertarget{il:rxs}{\texttt{rxs}} searches dictionary \emph{text objects} with regular expressions.\index{regular expressions!\texttt{rxs}}
Text objects are essentially all nonnouns.  Typical text objects are verbs, test cases, macros,  group and suite headers, 
long and short documentation texts, and dictionary documentation.
\texttt{rxs}'s arguments use JOD object and qualifier codes: see appendix~\ref{ap:objqualcodes} on
page~\pageref{ap:objqualcodes}.

\begin{wordhead}
\monad & rxs blclNames \\
\end{wordhead}
\begin{lstlisting}[frame=single,framerule=0pt]
  NB. all nonempty nonnoun search text
  rxs }. dnl ''  
  
  NB. nouns are excluded from searches
  rxs }. 0 1 0 dnl ''  
\end{lstlisting}  

\begin{wordhead}
\dyad & clRegex rxs blclNames \\
         & (clRegex ; iaOption) rxs zl \argsep blclNames \\
         & (clRegex ; ilOptions) rxs zl \argsep blclNames \\
\end{wordhead}
\begin{lstlisting}[frame=single,framerule=0pt]
  NB. empty patterns return nonempty nonnoun search text
  '' rxs }. dnl''

  NB. nonemtpy group and suite headers
  ('';2) rxs }. 2 dnl ''
  ('';3) rxs }. 3 dnl '' 

  NB. first match of  unique in nonnouns with names starting with re
  'unique' rxs }. dnl 're' 
   
  NB. first match of catch. in path words - note pattern escape
  ('catch\.';0 7 1) rxs }. dnl ''

  NB. words in group mytheory containing NIMP:
  'NIMP:' rxs }. grp 'mytheory'
  
  NB. all J quoted text in nonnoun path words starting with s
  ('''(''''|[^''])*''';0 7 2) rxs }. dnl 's'  

  NB. all digits in nonempty long word document text
  ('[[:digit:]]+';0 9 2) rxs }. dnl ''

  NB. digits in long word document text with match position arrays
  ('[[:digit:]]+';0 9 3) rxs }. dnl ''
  
  NB. many useful J regex patterns are in the jregex locale

  NB. all J quoted text with match position arrays 
  (Jchar_jregex_;0 7 3) rxs }. dnl 's'  
   
  NB. all J names in nonempty group header text
  (Jname_jregex_;2 7 2) rxs }. 2 dnl ''
  
  NB. first match of github in dictionary document
  ('github';5) rxs ''
  
  NB. all matches of github in dictionary document with position arrays
  ('github';5 7 3) rxs ''
\end{lstlisting}   
          

\subsection{\texttt{uses} --- return word uses} 

\hypertarget{il:uses}{\texttt{uses}} lists words used by other words. The lists 
are derived from the cross references\index{cross references!\texttt{uses}} stored 
by \hyperlink{il:globs}{\texttt{globs}}. The typical result
 of \texttt{uses} is a boxed table. Column 0 is a list of names and column 1 is list 
 of pairs of boxed lists. Each boxed list pair contains nonlocale 
 and locale\index{locale!references} global references. 

When computing the \emph{uses union}\index{uses union!\texttt{uses}}, (option 31), only nonlocale references\index{uses!references} are searched for further references. In general it is not possible to search locale references as they typically refer to objects created at runtime. In this system such references are treated as black boxes. It is important to know an object is being referenced even if you cannot peer inside the object.

\begin{wordhead}
\monad & uses blclName \argsep clName \\
\end{wordhead}
\begin{lstlisting}[frame=single,framerule=0pt] 
  NB. list all words used by words (0 globs) 
  uses ;:'word globals' 
\end{lstlisting}    

\begin{wordhead}
\dyad & iaObject uses blclName \argsep clname \\
\end{wordhead}
\begin{lstlisting}[frame=single,framerule=0pt] 
  NB. same as monad
  0 uses 'word'  
  
  NB. uses union of word 
  31 uses ;:'all known words we call'   
\end{lstlisting}  