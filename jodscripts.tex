
\section{JOD Scripts}

\subsection{Generated Script Structure}\label{ss:genscripts}

To use dictionary words it is necessary to generate scripts.  JOD scripts come in three flavors:
\begin{enumerate}
	\item Arbitrary J scripts 
  \item Header and list scripts
  \item Dump scripts\index{dump scripts}
\end{enumerate}

JOD test, macro and group/suite headers are arbitrary J scripts.  There are no restrictions on 
these scripts.  Group and suite scripts generated by 
\hyperlink{il:make}{\texttt{make}}, \hyperlink{il:mls}{\texttt{mls}} and 
\hyperlink{il:lg}{\texttt{lg}}, (see pages~\pageref{ss:make}, \pageref{ss:mls},
\pageref{ss:lg}), are header and list scripts.  \texttt{make} produces dump scripts.  

JOD script structure mirrors what you typically do in a J application script.  With most J application scripts you: 
\begin{enumerate}
	\item Setup the application's runtime environment.
	\item Load the classes, words and data that comprise the application.
	\item Start the application.
\end{enumerate}

This pattern of \emph{setup}, \emph{load} and \emph{start} is seen over and over in J scripts: see  
Table~\ref{tab:genscripts} on page~\pageref{tab:genscripts}.

\begin{table}[htbp]
  \centering
   \scriptsize
   \begin{tabular}{|l|l|p{1.5in}|p{3.0in}|} \hline
   \multicolumn{4}{|c|}{\textbf{\normalsize Generated Script Structure\T\B}} \\ \hline
   \multicolumn{1}{|c|}{\textbf{\normalsize Section\T}} & 
   \multicolumn{1}{|c|}{\textbf{\normalsize Type}} & 
   \multicolumn{1}{|c|}{\textbf{\normalsize Description}} & 
   \multicolumn{1}{|c|}{\textbf{\normalsize Example\B}} \\ \hline
    \textbf{\normalsize Setup} & \textsl{\normalsize Active} & \begin{minipage}{1.5in}
    Define group and Suite headers.  Headers 
    may contain one dependent section: see page~\pageref{ss:depsec}. \end{minipage}
     & \begin{minipage}{2.9in}
\begin{tabular}{l}
\textcolor{CodeComment}{\ttfamily\textsl{NB. define a group header}} \\
\verb|2 1 put 'groupname';' ... script text ... '| \\
\\
\textcolor{CodeComment}{\ttfamily\textsl{NB. define a suite header}} \\
\verb|3 1 put 'suitename';' ... suite text ... '| \\
\end{tabular} 
\end{minipage} \\ \hline
 \textbf{\normalsize Load} & \textsl{\normalsize Passive} & \begin{minipage}{1.5in}
   Load lists of words or tests.  \emph{Only word lists are passive.} Tests are typically active scripts. \end{minipage}
     & \begin{minipage}{2.9in}
\begin{tabular}{l}
\textcolor{CodeComment}{\ttfamily\textsl{NB. form group from stored words}} \\
\verb|grp 'groupname' ; ;:'words in group'| \\
\\
\textcolor{CodeComment}{\ttfamily\textsl{NB. form suite from stored tests}} \\
\verb|3 grp 'suitename' ; ;:'stored tests'| \\
\end{tabular} 
\end{minipage} \\ \hline
 \textbf{\normalsize Start} & \textsl{\normalsize Active} & \begin{minipage}{1.5in}
    Associate a postprocessor\index{postprocessor} macro with a group or suite. Postprocessors 
    are prefixed with \verb|POST_| \end{minipage}
     & \begin{minipage}{2.9in}
\begin{tabular}{l}
\textcolor{CodeComment}{\ttfamily\textsl{NB. group postprocessor}} \\
\verb|4 put 'POST_groupname';21;' ... script ... '| \\
\\
\textcolor{CodeComment}{\ttfamily\textsl{NB. test suite postprocessor}} \\
\verb|4 put 'POST_suitename';21;' ... script ... '| \\
\end{tabular} 
\end{minipage} \\ \hline
    \end{tabular}
   \caption{JOD generated script\index{scripts!generation!structure} structure}
  \label{tab:genscripts}
\end{table}


   
\subsection{Dependent Section}\label{ss:depsec}
   
A dependent section\index{dependent section} is a delimited subsection of a 
group or suite header, (see 
\hyperlink{il:grp}{\texttt{grp}} on page~\pageref{ss:grp}), that is used to define related words and runtime globals.  \emph{Global words defined in a dependent section are removed from group lists when groups are generated with \hyperlink{il:make}{\texttt{make}}, 
\hyperlink{il:mls}{\texttt{mls}} and \hyperlink{il:lg}{\texttt{lg}}: see pages~\pageref{ss:make},
\pageref{ss:mls} and~\pageref{ss:lg}}. This insures that the values assigned in the dependent section are maintained when the group script loads. 

A dependent section is delimited with \textcolor{CodeComment}{\texttt{\textsl{NB.*dependents}}} and  \textcolor{CodeComment}{\texttt{\textsl{NB.*enddependents}}} and only one dependent section per group header is allowed. The following is the dependent section in the \texttt{jod} class group header.  Globals in a dependent section are returned by \hyperlink{il:gdeps}{\texttt{gdeps}}, see page~\pageref{ss:gdeps}.

\begin{lstlisting}[frame=single,framerule=0pt,basicstyle=\ttfamily\footnotesize]
NB.*dependents x-- words defined in this section have related definitions
NB. host specific z locale nouns set during J profile loading
NB. (*)=: IFWIN UNAME
LF=:10{a.
CR=:13{a.
TAB=:9{a.
CRLF=:CR,LF
NB. option codes - to add more add a new object code
NB. and modify the following definition of MACROTYPE
JSCRIPT=:21
LATEX=:22
HTML=:23
XML=:24
TEXT=:25
BYTE=:26
UTF8=:28
NB. macro text types, depends on: JSCRIPT,LATEX,HTML,XML,TEXT,BYTE,UTF8
MACROTYPE=:JSCRIPT,LATEX,HTML,XML,TEXT,BYTE,UTF8
NB. object codes
WORD=:0
TEST=:1
GROUP=:2
SUITE=:3
MACRO=:4
NB. object name class, depends on: WORD,TEST,GROUP,SUITE,MACRO
OBJECTNC=:WORD,TEST,GROUP,SUITE,MACRO
NB. bad object code, depends on: OBJECTNC
badobj=:[: -. [: *./ [: , ] e. OBJECTNC"_
NB. path delimiter character & path punctuation characters
PATHDEL=: IFWIN { '/\'
PATHCHRS=:' :.-',PATHDEL
NB. default master profile user locations
JMASTER=:jodsystempath 'jmaster'
JODPROF=:jodsystempath 'jodprofile.ijs'
JODUSER=:jodsystempath 'joduserconfig.ijs'
NB.*enddependents
\end{lstlisting}

