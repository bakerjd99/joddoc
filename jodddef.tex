

\section{JOD Direct Definition Support}\label{ap:jodddef}

J version 9.02 (February 2021) introduced \index{direct definition} \emph{direct definitions}. The
designers of APL languages, of which J is a member, introduced direct definitions
long after \emph{canonical} or notorious ``$\pmb\triangledown$ editor'' style definitions. In retrospect,
most agree the entire family of languages would be better off if direct definition
had come first. Direct definitions are more elegant, concise, versatile, and beautiful than
clunky canonical equivalents. They're also easier to comprehend and compose than
long J tacit definitions. But history is history, and software history is hard to ignore because of the \emph{installed base}\footnote{Also known as \emph{users}}.
We're stuck with our kludges!

In JOD's case this shows up in how \hyperlink{il:globs}{\texttt{globs}}, see page~\pageref{ss:globs}, classifies global and local names in words
that contain direct definitions.  Consider the following:


\begin{lstlisting}[frame=single,framerule=0pt,basicstyle=\ttfamily\footnotesize]
make_my_def=: 3 : 0

NB. local to make_my_def
here=. we=. 2 + are=. 3 * local=. 4

global_adv=: {{*./"1 u/&> 2 <\"1 y}}

local_verb=. {{)d
  NB. not local to make_my_def
  we=. are=. not=. make_my_def=. local=. x
  if. 1-:y do.
    NB. deep global to make_my_def
    {{deep_gbl=: 'deep';":y}} y  
  end.
}}

0 local_verb y
)
\end{lstlisting}

\texttt{make\_my\_def} contains local and global direct definitions. When executed it creates  \texttt{global\_adv}
and runs \texttt{local\_verb} which in turn creates \texttt{deep\_gbl}.  When \texttt{local\_verb} runs it creates
a local namespace like any explicit J verb. Hence \texttt{not} and \texttt{make\_my\_def}  are not \emph{strictly} \texttt{make\_my\_def} locals.  \emph{The
execution of  directly defined local words is equivalent to calling explicit words.}  \texttt{globs} does not track direct definition name scopes. 
It views all direct definition names as if they belonged to the topmost word.  For \texttt{make\_my\_def}  \texttt{globs} this gives:

\begin{lstlisting}[frame=single,framerule=0pt,basicstyle=\ttfamily\footnotesize]   
   NB. JOD name classification
   11 globs 'make_my_def' 
\end{lstlisting}

\newpage

\begin{lstlisting}[frame=single,framerule=0pt,basicstyle=\ttfamily\footnotesize]   
+-+-------------------------------------------------------+
|1|+------+----------------------------------------------+|
| ||Global|+--------+----------+                         ||
| ||      ||deep_gbl|global_adv|                         ||
| ||      |+--------+----------+                         ||
| |+------+----------------------------------------------+|
| ||Local |+---+----+-----+----------+-----------+---+--+||
| ||      ||are|here|local|local_verb|make_my_def|not|we|||
| ||      |+---+----+-----+----------+-----------+---+--+||
| |+------+----------------------------------------------+|
| ||(*)=: |                                              ||
| |+------+----------------------------------------------+|
| ||(*)=. |                                              ||
| |+------+----------------------------------------------+|
| ||for.  |                                              ||
| |+------+----------------------------------------------+|
+-+-------------------------------------------------------+
   
   NB. execute and show name classes
   make_my_def 1
+----+-+
|deep|1|
+----+-+
   nc ;:'local_verb global_adv deep_gbl' 
_1 1 0
\end{lstlisting}

When embedding direct defintions in explicit words, or within other direct definitions, it's a good idea to make names distinct.
For example, in embedded \texttt{local\_verb} do something like:

\begin{lstlisting}[frame=single,framerule=0pt,basicstyle=\ttfamily\footnotesize]   
local_verb=. {{)d
  weLv=. areLv=. notLv=. make_my_defLv=. localLv=. x
  if. 1-:y do.
    NB. deep global to make_my_def
    {{deep_gbl=: 'deep';":y}} y  
  end.
}}
\end{lstlisting}









