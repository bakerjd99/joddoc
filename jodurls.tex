% urls embedded in document source.  This contents of this file
% has been derived from the output of J word (extracthrefs}
     
\begin{description}
                
\item  \emph{The JOD Pages}.\index{URL}  This website maintains references to all
JOD related documents and downloads.

\jodurl{http://bakerjd99.googlepages.com/home}
                                 
\item Wikipedia entry for \emph{Hungarian Notation}:  it's tedious and overwrought
but conveys the essentials.

\jodurl{http://en.wikipedia.org/wiki/Hungarian_notation}
                       
\item The website of the legendary computer scientist Donald Knuth.  Knuth created
the typesetting language \TeX\ in the
1970's and \TeX\ is still in widespread use because nothing
developed since is any better.  Genius is hard to replace!

\jodurl{http://www-cs-faculty.stanford.edu/~knuth/}
                            
\item All about the \emph{Scheme} programming language. 

 \jodurl{http://www-swiss.ai.mit.edu/projects/scheme/}
                          
\item \emph{Common Lisp} is the industrial strength version of the \texttt{LISP}
family of programming languages.  It's star has been waning since
the late 1980's.  

\jodurl{http://www.cs.cmu.edu/Groups/AI/html/cltl/cltl2.html}
                  
\item The  website for \emph{Graphviz}.  Graphviz is an amazing open source
system that draws graphs.  Some of the diagrams in this document were
generated with Graphviz using the J Graphviz addon.

\jodurl{http://www.graphviz.org}
                                               
\item J documentation for the \texttt{scriptdoc} utility.  

\jodurl{http://www.jsoftware.com/help/user/scriptdoc.htm}
                      
\item J Wiki download page for the \texttt{jodsource} addon.\index{Addon}

\jodurl{http://www.jsoftware.com/jwiki/Addons/general/jodsource}
               
\item J Wiki download page for the \texttt{jod} addon.

\jodurl{http://www.jsoftware.com/jwiki/Addons/general/jod}

\item J Wiki download page for the \texttt{graphviz} addon. After JOD this is
my favorite J addon. Oleg Kobchenko has created a jewel for J users!

\jodurl{http://www.jsoftware.com/jwiki/Addons/graphics/graphviz}

\item Oleg's J Page.  Chock full of interesting examples of J programming.

\jodurl{http://olegykj.sourceforge.net/}
                     
\item J Wiki addon list.  This is a complete list of J addons.

\jodurl{http://www.jsoftware.com/jwiki/Addons}
                                 
\item J Wiki documentation about JAL: the J package manager.  JAL is
the main tool for downloading and installing J addons. 

\jodurl{http://www.jsoftware.com/jwiki/JAL/Package_Manager}
                                              
\item Main J page.  This is were you go to download the latest version of J.  

\jodurl{http://www.jsoftware.com}
                                              
\item A hard copy spiral bound book version of this document is available 
here.  The price of the book is at cost. 

\jodurl{http://www.lulu.com/content/3229961}

\end{description}
