\section{Best Practices}

Here are some JOD practices I have found useful.  A JOD lab\index{labs}, \href{https://github.com/bakerjd99/jod/blob/master/jodnotebooks/JODBestPracticesLab.pdf}{\emph{JOD (3) Best Practices}}, elaborates and reiterates this material.  After 
reading these notes I recommend you run this lab.  JOD labs are 
found in the \texttt{Addons} lab category.\footnote{Prior J versions put JOD labs in the \texttt{General} category.}


\begin{description}
\item[JOD does not belong in the J tree.] Never store your JOD dictionaries in J install directories!  Create a JOD master dictionary directory root that is  independent of J: see \hyperlink{il:newd}{\texttt{newd}} on page~\pageref{ss:newd}.  It's also a good idea to define a subdirectory structure that mirrors J's versions. 

\begin{lstlisting}[frame=single,framerule=0pt]
NB. create a master JOD directory root outside of J's directories.
newd 'bptest';'c:/jodlabs/j701/bptest';'best practices dictionary'

NB. linux and mac paths start with a leading /
newd 'bptest';'/home/john/jodlabs/bptest'
\end{lstlisting}

\item[Back up, back up, and then back up.]  It's easy to back up with JOD so back up often: see 
\hyperlink{il:packd}{\texttt{packd}} and \hyperlink{il:restd}{\texttt{restd}} on pages~\pageref{ss:packd} and ~\pageref{ss:restd}.

\begin{lstlisting}[frame=single,framerule=0pt]
NB. open the best practice dictionary
od 'bptest' [ 3 od ''

NB. back it up
packd 'bptest'
\end{lstlisting}
 

\item[Take a script dump.]  It's a good idea to ``dump'' your dictionaries as plain text.  JOD can dump all open dictionaries as a single J script: see \hyperlink{il:make}{\texttt{make}} on page~\pageref{ss:make}. Script dumps are the most stable way to store J dictionaries.  The 
\href{https://www.jsoftware.com/jwiki/Addons/general/jodsource}{\texttt{jodsource}} addon 
distributes all JOD source code in this form.

\begin{lstlisting}[frame=single,framerule=0pt]
NB. (make) creates a dictionary dump in the dump subdirectory
make ''
\end{lstlisting}

Dump scripts are essential JOD dictionary maintenance tools: see appendix~\ref{ap:joddumptricks} on page~\pageref{ap:joddumptricks}.
 
\item[Make a master re-register script.]   JOD only sees dictionaries registered in the \verb|jmaster.ijf| file:
see Table~\ref{tab:jmaster} on page~\pageref{tab:jmaster}.
Maintaining a list of registered dictionaries is recommended.  JOD can generate a re-register script: see \hyperlink{il:od}{\texttt{od}} on page~\pageref{ss:od}.  Generate a re-register script and put it in your main JOD dictionary directory root.

\begin{lstlisting}[frame=single,framerule=0pt]
NB. generate re-register script
rereg=:  ;{: 5 od ''
\end{lstlisting}
 

\item[Set library dictionaries to \texttt{READONLY}.]  Open JOD dictionaries define a search path.  The first dictionary on the path is the only dictionary that can be changed.  
It is called\index{put dictionary} the \emph{put dictionary.}  Even though nonput 
dictionaries cannot be changed by JOD it's a good idea to set them \texttt{READONLY} because:
\begin{enumerate}
\item \texttt{READONLY} dictionaries can be accessed by any number of JOD tasks. \texttt{READWRITE} dictionaries can only be accessed by one task.  
\item Keeping libraries \texttt{READONLY} prevents accidental \texttt{put}'s as you open and close dictionaries.
\end{enumerate}

\begin{lstlisting}[frame=single,framerule=0pt]
NB. make bptest READONLY
od 'bptest' [ 3 od ''
dpset 'READONLY'
\end{lstlisting}
 
 
\item[Keep references updated.]  JOD stores word references: see \hyperlink{il:globs}{\texttt{globs}} on page~\pageref{ss:globs}.  References enable many useful operations. References allow \hyperlink{il:getrx}{\texttt{getrx}}, see page~\pageref{ss:getrx}, to load words that call other words in new contexts.

\begin{lstlisting}[frame=single,framerule=0pt]
NB. only put dictionary references need updating
0 globs&> }. revo ''
\end{lstlisting}
 
 
\item[Document dictionary objects.]  Documentation is a long standing sore point for programmers.  
Most of them hate it. Some claim it's unnecessary and distracting.  Many put in half-assed efforts. 
In my opinion this is ``not even wrong!''  Good documentation elevates code. In 
\href{https://www-cs-faculty.stanford.edu/~knuth/}{Knuth's} \cite{knuth:web} opinion it separates
``literate programming'' from the odious alternative.   JOD provides a number of easy ways to document code: see \hyperlink{il:doc}{\texttt{doc}} on page~\pageref{ss:doc}, 
\hyperlink{il:put}{\texttt{put}} on page~\pageref{ss:put} and \hyperlink{il:nw}{\texttt{nw}} on page~\pageref{ss:nw}.


\item[Define your own JOD shortcuts.]  JOD words can be used 
within arbitrary J programs.  If you don't find a JOD primitive that meets 
your needs do a little programming.

\begin{lstlisting}[frame=single,framerule=0pt]
NB. describe a JOD group - (ijod) is JOD's interface locale
hg_ijod_=: [: hlpnl [: }. grp
   
NB. re-reference put dictionary show any errors
reref_ijod_=: 3 : '(n,.s) #~ -.;0{"1 s=.0 globs&>n=.}.revo'''' [ y'
   
NB. words referenced by group words that are not in the group
jodg_ijod_=: 'agroup'
nx_ijod_=: 3 : '(allrefs  }. gn) -. gn=. grp jodg'
   
NB. missing from (agroup)
nx ''
\end{lstlisting}
 

\item[Customize JOD edit facilities.]  The main JOD edit words \hyperlink{il:nw}{\texttt{nw}} on page~\pageref{ss:nw} and
 \hyperlink{il:ed}{\texttt{ed}} on page~\pageref{ss:ed} can be customized by defining a \texttt{DOCUMENTCOMMAND} script.

\begin{lstlisting}[frame=single,framerule=0pt]
NB. define document command script - {~N~} is word name placeholder
DOCUMENTCOMMAND_ijod_ =: 0 : 0
smoutput pr '{~N~}'
) 
   
NB. edit a new word - opens edit window
nw 'bpword'
\end{lstlisting}
 

\item[Define JOD project macros.]  When programming with JOD you typically 
open dictionaries, load system scripts and define nouns. This can be done in 
a project macro script: see \hyperlink{il:rm}{\texttt{rm}} on page~\pageref{ss:rm}.

\begin{lstlisting}[frame=single,framerule=0pt]
NB. define a project macro - I use the prefix prj for such scripts
prjsunmoon=: 0 : 0

NB. standard j scripts
require 'debug task'

NB. local script nouns 
jodg_ijod_=: 'sunmoon'
jods_ijod_=: 'sunmoontests'

NB. put/xref
DOCUMENTCOMMAND_ijod_ =: 'smoutput pr ''{~N~}'''
)
   
NB. store macro
4 put 'prjsunmoon';JSCRIPT_ajod_;prjsunmoon

NB. setup project
rm 'prjsunmoon'  [ od ;:'bpcopy bptest' [ 3 od ''
\end{lstlisting}
 
 
\item[Maintain a make load scripts macro or test.] To simplify the maintenance of JOD generated load scripts
create a macro or test script that rebuilds load scripts when executed with \hyperlink{il:rm}{\texttt{rm}} on 
page~\pageref{ss:rm} or \hyperlink{il:rtt}{\texttt{rtt}} on page~\pageref{ss:rtt} . 

\begin{lstlisting}[frame=single,framerule=0pt]
NB.*make_load_scripts s-- generates load scripts.
NB.
NB. Can run as a tautology test see: rtt 

cocurrent 'base'
require 'general/jod'
coclass 'AAAmake999' [ coerase <'AAAmake999'

>0{OPENDIC=: did 0
>0{od 'utils' [ 3 od ''  

NB. utils scripts
>0{mls 'bstats'
>0{mls 'xmlutils'
>0{3 od ''

NB. jod tester 
>0{od ;:'joddev jod utils'
>0{mls 'jodtester'
>0{3 od ''

NB. exif image utils 
>0{od ;:'smugdev smug image utils'
>0{mls 'exif'
>0{3 od ''

cocurrent 'AAAmake999'
>0{od }. OPENDIC

cocurrent 'base'
coerase <'AAAmake999'
\end{lstlisting}

 
\item[Edit your \texttt{jodprofile.ijs}.]  When JOD loads the profile script
\begin{verbatim}
 ~addons/general/jod/jodprofile.ijs
\end{verbatim}
is run: see appendix~\ref{ap:jodprofile} on page~\pageref{ap:jodprofile}.  Use this 
script to customize JOD. Note how you can execute project macros when JOD loads with sentences like:

\begin{lstlisting}[frame=single,framerule=0pt]
NB. open required dictionaries and run project macro
rm 'prjsunmoon' [ od ;:'bpcopy bptest'
\end{lstlisting}
 

\item[Use JOD documentation.] JOD documentation is available as a
PDF document \verb|jod.pdf|
that can be read with \hyperlink{il:jodhelp}{\texttt{jodhelp}}: see page~\pageref{ss:jodhelp}.

\begin{lstlisting}[frame=single,framerule=0pt]
NB. display JOD help - requires general/joddocument addon
jodhelp 0
\end{lstlisting}

\end{description}